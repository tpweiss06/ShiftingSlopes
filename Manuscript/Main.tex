\documentclass[12pt, oneside]{article}
\usepackage{geometry}                		
\geometry{letterpaper}                   		
\usepackage{graphicx}
\usepackage{graphics}
\usepackage{hyperref}
\usepackage{fancybox}
\usepackage[centertags]{amsmath}
\usepackage{amssymb}
\usepackage{amsthm}
\usepackage{natbib}				
\usepackage{fullpage}
\usepackage{placeins}
\usepackage{setspace}
\usepackage{lineno}
\usepackage{color}
\usepackage{mathptmx}
\usepackage{authblk}

\DeclareRobustCommand{\firstsecond}[2]{#1}

\newcommand{\mb}{\mathbf}
\newcommand{\bs}{\boldsymbol}
\newcommand{\wt}{\widetilde}
\newcommand{\s}{^{(s)}}

\title{(1) Local adaptation and dispersal evolution interact to drive population response to climate change \\ (2) Spatial population structure drives extinction dynamics in climate-induced range shifts \\ (3) Initial spatial structure drives population response to climate change}

\date{}

\author[1,*]{Christopher Weiss-Lehman}
\author[1]{Allison K. Shaw}


\affil[1]{Ecology, Evolution, and Behavior, University of Minnesota}
\affil[*]{Denotes corresponding author}

\begin{document}
\maketitle

\begin{flushleft}
Corresponding author details: \\
Christopher Weiss-Lehman \\
email: cweissle@umn.edu \\
phone: +1 (720) 590 2278 \\
\end{flushleft}

\doublespacing
\linenumbers

\newpage

\section*{Abstract}
Anthropogenic climate change has already caused some species' ranges to shift upwards in latitude and elevation, making it imperative to understand the ecological and evolutionary dynamics behind such range shifts. While insights can be gained from the related process of range expansion, climate-induced range shifts have several key attributes differentiating them from range expansions. In particular, range shifting populations are characterized by some initial spatial population structure defined by factors such as local adaptation and the nature of the range edge. Here, we use an individual-based model to explore the importance of spatial population structure for range shifting populations. We show that populations with more pronounced spatial structure have substantially increased extinction probabilities. Further, in our simulations evolution of dispersal ability was insufficient to rescue faltering populations. Rather, a population's fate during climate change was determined by the evolved composition of dispersal phenotypes within the initial range; only populations consisting of highly dispersive individuals prior to the onset of climate change survived. Our results demonstrate that dispersal evolution alone may be insufficient to save a range shifting population and that the initial spatial population structure plays a pivotal role in determining the outcome of climate-induced range shifts.

\begin{flushleft}
\textbf{Key words:} range shifts, eco-evolutionary dynamics, local adaptation, individual-based model
\end{flushleft}

\newpage

\section*{Introduction}
Climate change is expected to dramatically reshape global biogeographic patterns as some species shift their ranges to track changing environmental conditions~\citep{gonzalez2010global}. These range shifts are generally predicted to proceed upwards in latitude, elevation, or both as average global temperatures continue to rise~\citep{loarie2009velocity}. In fact, contemporary range shifts have already been observed in a wide range of taxa~\citep{chen2011rapid, parmesan2006ecological}. Such range shifts present significant challenges to current and future conservation efforts as they can result in the extinction of populations failing to track a changing climate~\citep{parmesan2006ecological} the creation of novel species assemblages as not all species shift their ranges at the same rate or at all~\citep{hobbs2009novel}, or both. It is therefore crucial to understand the dynamics of such climate-induced range shifts to better inform current and future conservation work.

While the study of range shifts due to climate change is relatively new, important insights can be gained from the related but distinct process of range expansion. Range expansions have been studied for decades, leading to a robust understanding of both the ecological~\citep{hastings2005spatial} and evolutionary~\citep{shine2011evolutionary, excoffier2009genetic} mechanisms responsible for shaping such expansions. For example, while the speed of a range expansion can be well approximated by a combination of the species' intrinsic growth rate and dispersal ability~\citep{hastings2005spatial}, recent research demonstrates that evolution in both of these traits can have important implications for both the mean and variance of expansion speed through time~\citep{weiss2017rapid, ochocki2017rapid, szHucs2017rapid, shaw2015dispersal}. Since range shifts, like range expansions, have a leading edge of population advance, they are likely to be affected by similar ecological and evolutionary mechanisms as have been shown to drive dynamics in range expansions~\citep{hargreaves2014evolution}. However, while such insights from range expansions are valuable for understanding and predicting dynamics of range shifts, it is important to recognize that these two processes have significant differences as well.

In particular, range expansions and range shifts due to climate change involve fundamentally distinct initial conditions. Range expansions, especially for invasive and reintroduced species, typically begin from the successful establishment and subsequent spread of a small, founding population~\citep{hastings2005spatial}. Such founding populations often represent samples from some larger source population and as such lack any initial spatial population structure. In contrast, range shifts start from entire populations with existing spatial structure within the initially stable range~\citep{hargreaves2014evolution}. Such spatial structure can manifest through local adaptation within the range, the severity of the gradient in population size from the range core to edge, or some combination of the two~\citep{hargreaves2014evolution, hargreaves2015fitness, henry2013eco}. 

These different aspects of spatial population structure have been shown to dramatically impact the response of populations to climate change. For example, the mechanism responsible for the gradient in population size from core to edge (e.g. declines in per capita growth rate vs. increases in local extinction probability) can directly change the probability of extinction a species faces during a climate driven range shift~\citep{henry2013eco}. A population's risk of extinction during a range shift has also been related to the degree of local adaptation within the range. Specifically, a broad environmental niche (i.e. little local adaptation) can decrease a population's ability to track a changing climate if dispersal occurs in a stepping stone manner, allowing some individuals to block dispersal of better adapted genotypes~\citep{atkins2010local}. Local adaptation also has the potential to interact with dispersal evolution during climate change, driving increased dispersal probability in an asexual species as genotypes shift to keep pace with their environmental optimum~\citep{hargreaves2015fitness}. However, it is unclear how dispersal evolution and local adaptation might interact in a sexually reproducing species in which dispersal and local adaptation are directly linked via gene flow. For example, evolution of increased dispersal could simultaneously reduce local adaptation within a population due to increased gene flow throughout the range. In fact, long-distance pollen dispersal in flowering plants has been shown to restrict local adaptation and, when pollen dispersal sufficiently outpaces seed dispersal, to lead to ecological niche shifts, rather than spatial range shifts, in response to simulated climate change~\citep{aguilee2016pollen}. Additionally, the distribution of dispersal phenotypes within a population can be influenced by the severity of the gradient at the range edge~\citep{henry2013eco, hargreaves2014evolution}, further complicating the relationship between dispersal evolution and local adaptation during climate-induced range shifts. Evolution during range shifts due to local adaptation and dispersal evolution are each potentially important drivers of range shift dynamics and it is therefore important to consider both when predicting the dynamics of populations responding to climate change.

Here, we develop a two-sex individual-based model with two genetically determined traits, one defining an individual's expected dispersal distance and the other determining an individual's environmental niche. Using this model, we vary the degree of spatial population structure defining the stable range to ascertain how this impacts a population's ability to track a changing climate. Further, we contrast the dynamics of extant and extinct populations to understand the joint impact of dispersal evolution and local adaptation on extinction risk during climate-induced range shifts.

\section*{Methods}
A full description of the individual-based model using the Overview, Design concepts, and Details protocol~\citep{grimm2010odd} is available in the online supplemental materials, but we present a brief summary here. Each simulated landscapes consisted of a discrete lattice of habitat patches with a fixed width in the $y$ dimension and unbounded length in the $x$ dimension. Environmental conditions varied along the $x$ dimension and remained constant across the $y$ dimension (Fig. S1). To simulate climate change, environmental conditions shifted a constant amount in the $x$ dimension each generation. Competition and reproduction occurred within each patch and local populations were linked via dispersal, assuming wrapping boundaries in the $y$ dimension to prevent edge effects. 

Individuals were characterized by two traits (dispersal and an environmental niche), both defined by a set of $5$ quantitative diploid loci, assuming no linkage. The dispersal trait defined an individual's expected dispersal distance, assuming an exponential dispersal kernel. An individual's environmental niche trait allowed for local adaptation; the closer the niche value to the environmental optimum of the individual's location, the higher the individual's realized fitness. The potential for local adaptation was then manipulated by varying the degree to which the environmental optimum changed throughout the landscape. In addition to the potential for local adaptation, simulated ranges were characterized by a decline in patch carrying capacity from the range center to the edge, the severity of which could be adjusted without altering the total carrying capacity of the landscape (see supplemental materials; Fig. S1). Reproduction within each patch occurred via a stochastic implementation of the classic Ricker model~\citep{ricker1954stock, melbourne2008extinction}. Using this model, we simulated non-overlapping generations consisting of discrete dispersal and reproduction phases (Fig. S2) to determine the effect of different aspects of spatial population structure on the eco-evolutionary dynamics of climate-induced range shifts.

Most model parameters were held constant over all simulations (Table S1), but we varied certain parameter values to explore the interacting roles of local adaptation and the severity of the gradient in environmentally suitable habitat at the range edge (Table S2). Specifically, we considered a factorial combination of no, low, and high potential for local adaptation with shallow, moderate, and stark gradients in suitable habitat at the range edge for a total of $9$ different scenarios, each explored with $200$ simulations. Each simulation ran for $2150$ generations with stable climate conditions for the first $2000$, followed by $100$ generations of climate change and a final $50$ generations of stable conditions. Figure~\ref{fig:SimExample} shows an example of a single simulation undergoing climate change. Additionally, we explored the role of the rate of climate change on the dynamics of range shifts in these different scenarios, examining both slower and faster speeds of climate change (see the online supplemental materials). 

For each parameter combination, we calculated the per patch change in mean dispersal phenotype from the beginning of the period of climate change to the end to quantify dispersal evolution. For this analysis, we defined individual patches by their relative location within the range rather than with their fixed spatial coordinates. In this way, we calculated change in mean dispersal phenotype for populations consistently occupying the same position relative to the range center (e.g. leading edge vs. core populations). Due to local extinctions, not all patches were occupied at the end of the period of climate change. To quantify dispersal evolution in these patches, we used data from the last generation in which the population had at least $10$ individuals as the end point to calculate change in the mean dispersal phenotype. Changes in mean dispersal phenotype were calculated by subtracting the initial mean dispersal phenotype from the value at the end of climate change (or at the last generation of at least $10$ individuals occupying the patch in the case of population extinctions) so that positive values indicate an increase in the mean dispersal phenotype. All simulations and data processing were performed in R version $3.4.4$~\citep{team2000r} and the code is available at \url{https://github.com/tpweiss06/ShiftingSlopes}.

\section*{Results}
For almost all parameter combinations examined here, a portion of the simulated populations were able to keep pace with changing climate conditions. Successful populations shifted their spatial distributions at essentially the same rate as climate change (e.g. Fig. 1). However, some populations failed to keep pace, lagging further behind the shifting climate until they went extinct. Populations defined by a higher potential for local adaptation and by stark habitat gradients at the range edge experienced the greatest probability of extinction due to climate change (quantified by the proportion of simulated populations to go extinct through time; Fig. 2). While both aspects of a population's range influenced extinction probabilities, the potential for local adaptation drove more dramatic changes to extinction dynamics. The parameters defining both of these range attributes were varied widely (the potential for local adaptation doubled from the low to high scenario and the parameter defining the severity of the environmental gradient was increased by a factor of 100 from shallow to stark gradients; Table S2), suggesting that potential for local adaptation may be the stronger driver of extinction risk during climate-induced range shifts across a wide region of parameter space. Additionally, the pace of climate change also influenced extinction probabilities as expected with faster climate change corresponding to greater extinction risk (Fig. S3 \& S4). However, this effect was independent of the roles of local adaptation and the habitat gradient at the range edge in determining the extinction probability for a range shifting population.

Dispersal evolution is expected to aid populations shifting spatially to track a changing climate, and, as expected, the average dispersal phenotypes increased through time in some simulations (e.g. Fig.~\ref{fig:SimExample}). However, comparing patterns in dispersal evolution between simulated populations that successfully tracked climate change and those that ultimately went extinct revealed no differences in the magnitude or direction of dispersal evolution (Fig. 3a\&b). Populations in all parameter combinations experienced both increases and decreases in average dispersal phenotypes, with all distributions of observed changes to dispersal phenotypes strongly centered on $0$ (Fig. S5-S7). The similarity in evolved changes to dispersal between surviving and extinct simulated populations suggests that this cannot explain the different outcomes achieved by these populations.

Instead, the initial distribution of dispersal phenotypes immediately before the beginning of climate change played a large role in determining the ultimate fate of simulated populations. Simulated populations evolved a range of dispersal phenotypes over the $2000$ generations of stable climatic conditions in response to the potential for local adaptation and the severity of the habitat gradient at the range edge (Fig. S8-S10). The populations that survived climate change were those that happened to make up the upper end of these distributions, containing individuals with high dispersal phenotypes (Fig. 3c\&d). In fact, comparing dispersal phenotypes of successful populations across different parameter combinations revealed a threshold in dispersal phenotypes separating individuals from successful and extinct populations (Fig. S8-S10). This threshold varies directly with the speed of climate change, increasing with more rapid climate change. Across all speeds of climate change, this threshold corresponds to roughly similar probabilities ($12\%$, $9\%$, and $7\%$ at slow, moderate, and fast speeds respectively) of an individual keeping pace with climate change (i.e. moving the correct number of patches in the $x$ dimension to track the changing environmental conditions).

Intriguingly, the simulations that survived climate change tended to be characterized by reduced fitness at the range margins compared to simulated populations that went extinct (Fig. S11-S13). This pattern was most evident in the simulations with either (1) a gradual environmental gradient at the range edge or (2) a high potential for local adaptation. Intuitively, there was no spatial pattern in fitness for simulations with no potential for local adaptation. As the simulations that survived climate change were also characterized by heightened dispersal (Fig. 3), the observed reduction in average patch fitness at the margins is likely due to increased gene flow from the range core hampering the abilities of these edge populations to adapt to local conditions.

\section*{Discussion}
Range shifts due to climate change represent a global threat to biodiversity and much recent research has focused on exploring the underlying ecological and evolutionary dynamics of such range shifts to inform conservation efforts. Here we developed an individual-based model to explore the eco-evolutionary dynamics of climate-induced range shifts in sexually reproducing, diploid populations with both dispersal and environmental niche traits defined by multiple loci. In contrast, previous models have focused on a subset of these factors: ecological dynamics (e.g.~\citep{brooker2007modelling}), evolution in a single trait only (e.g.~\citep{atkins2010local, henry2013eco}), and relatively simple genetic scenarios (e.g. single-locus haploid genetics in asexual populations~\citep{boeye2013more, hargreaves2015fitness}). Thus, in analyzing our model, we tested the generality of previous results under realistic levels of biological complexity. Specifically, we demonstrated the role of spatial population structure, in the form of local adaptation and the environmental gradient defining the range edge, in determining extinction risk for range shifting populations via impacts on the initial distribution of dispersal phenotypes and environmental niche values.

Our results suggest that populations most likely to keep pace with climate change will be those with little to no local adaptation within the range and in locations with shallow environmental gradients defining the range edge (Fig. 2). A survey of the scientific literature found evidence for local adaption in approximately $71\%$ of studies testing for it, suggesting a high prevalence of local adaptation in natural populations~\citep{hereford2009quantitative}. Further, a recent meta-analysis of $1400$ bird, mammal, fish, and tree species found no evidence for consistent declines in abundance towards range edges~\citep{dallas2017species}, suggesting many species exhibit similar abundances at the edge and center of their ranges similar to the stark environmental gradients imposed in our study. While some of these patterns could represent a publication bias, for example against negative results in studies of local adaptation, combined with our results they suggest many species will face elevated extinction risks in climate-induced range shifts due to the spatial population structure of their initial ranges.

A key result of our model was the importance of the initial distribution of dispersal phenotypes (i.e. prior to the start of climate change) in determining a population's extinction risk during climate change (Fig. 3c\&d). Survival in the face of climate change was primarily determined by whether or not simulated populations consisted of individuals with dispersal phenotypes at or above some threshold value. Importantly, the threshold necessary to survive climate change itself was constant across all parameter combinations at a given speed of climate change (Fig. S8-S10). The differences in extinction probabilities among different scenarios were instead driven by the overall distribution of dispersal phenotypes that evolved in different simulations under each parameter combination. Scenarios with no potential for local adaptation and gradual environmental gradients simply had larger proportions of high dispersal phenotypes under stable climate conditions. A high potential for local adaptation, in contrast, selected against such high dispersal phenotypes due to dispersal's homogenizing effect on population genetic structure~\citep{lenormand2002gene}. Similarly, a more severe habitat gradient at the range edge increases the risk of dispersing beyond the boundary of suitable habitat, resulting in selection against heightened dispersal~\citep{shaw2014population}. Thus, attributes defining the spatial population structure of simulated ranges resulted in different initial distributions of traits under stable climate conditions, which then determined the extinction risk of populations under climate change.

While high dispersal phenotypes prior to climate change allowed populations to survive, it had the additional effect of reducing average fitness at the range margins in scenarios with the potential for local adaptation (Fig. S11-S13). In the model, populations at the range margins tended to be lower abundance than populations in the range core, thus increasing their susceptibility to gene flow from the core~\citep{lenormand2002gene}. Thus, when populations were defined by high dispersal phenotypes prior to climate change, it reduced fitness at the range margin due to gene flow preventing adaptation to local conditions. As a result, the populations most likely to survive climate change were, counterintuitively, also those characterized by lower fitness prior to the start of climate change. While not all populations are characterized by small populations at the range edges~\citep{dallas2017species}, and thus are unlikely to be equally susceptible to the influence of gene flow from the core, our results suggest that populations exhibiting high levels of local adaptation within their stable range are likely to be at greater risk of extinction during periods of climate change.

Previous research has suggested that evolution of increased dispersal ability during climate change may be able to rescue populations that would otherwise be unable to keep pace with shifting environmental conditions~\citep{boeye2013more}. Our results suggest this is not always the case, and in fact may only be possible under certain, relatively narrow conditions. Models finding that dispersal evolution may rescue populations during climate change have typically used relatively simple genetic frameworks to model dispersal, including haploid and single-locus models~\citep{boeye2013more, hargreaves2015fitness}. As dispersal evolution during range expansions and shifts occurs via spatial sorting~\citep{shine2011evolutionary}, such simplified genetic frameworks may allow more efficient sorting of highly dispersive alleles. Additionally, many previous models of climate-induced range shifts use asexual reproduction which can result in more rapid rates of evolutionary change under certain conditions~\citep{crow1965evolution, smith1968evolution}. The negligible role played by dispersal evolution in our model (Fig. 3a\&b) suggests that when such simplifying assumptions are relaxed the potential for population rescue via evolution of heightened dispersal is greatly reduced, thus increasing the role of the initial spatial population structure within the range in determining a population's fate under climate change.

\subsection*{Conclusion}
Understanding the various ecological and evolutionary drivers of climate-induced range shifts is crucial to current and future conservation efforts. In particular, we need to better understand how these various mechanisms combine to shape the extinction probabilities of populations undergoing range shifts, which, in turn, will allow more focused interventions. Our results suggest that initial spatial population structure, as determined by local adaptation and the environmental gradient at the range edge, has the potential to dramatically alter the extinction probability faced by species responding to climate change. Further, in contrast to other studies assuming more simplified breeding and genetic structure, we find very little role for the evolution of heightened dispersal abilities to rescue a population composed of low-dispersing individuals. Future work should continue to examine the interplay between initial conditions in range shifts and the potential for evolutionary rescue. As climate change continues to accelerate~\citep{chen2017increasing}, it is imperative to not only identify those factors leading to increased extinction risk in range shifting populations, but also to develop meaningful conservation strategies to mitigate such increased risk.

\section*{Acknowledgements}
We thank Lauren Shoemaker, Lauren Sullivan, Kate Meyers, Dongmin Kim, Lindsay Leverett, and members of the Theory Under Construction group at the University of Minnesota for providing thoughtful comments on the manuscript. 

\bibliographystyle{ecology}
\bibliography{main_bib}

\newpage

\section*{Figure legends}
\paragraph{Figure 1.} A single example of a simulation with a high potential for local adaptation and a moderate habitat gradient defining the range edge. Information on the abundance, dispersal, and fitness of each patch in the population is shown for time periods beginning with the last generation of stable climate conditions ($t = 0$) to 40 generations after the start of climate change. Log transformed mean dispersal phenotypes are shown for each patch. Average patch fitness was calculated based on the mean local environmental niche phenotype and the optimal niche value for that location.

\paragraph{Figure 2.} The cumulative probability of extinction due to climate change in different experimental scenarios. Graphs show the proportion of simulated populations that went extinct through time for scenarios with (a) no, (b) low, and (c) high potential for local adaptation, and in environments characterized by a shallow (solid line), moderate (dashed line), or stark (dotted line) gradient at the range edge.

\paragraph{Figure 3.} Patterns in the evolution and the initial distribution of the dispersal trait, highlighting extant simulations. Evolution in dispersal (a and b) is shown as the change in the mean dispersal phenotype of each patch from the beginning of the period of climate change to the end. Positive values indicate an increase in average dispersal ability in the patch. Initial distributions of the dispersal trait (c and d) are shown with histograms of the log transformed dispersal phenotypes of individuals in populations after $2000$ generations of stable climate conditions. In all panels, values associated with extant populations are shown in dark blue. Results are shown for populations with no potential for local adaptation and a gradual environmental gradient at the range boundary (a and c; $n = 155$ extant populations) and for populations with a high potential for local adaptation and a stark gradient at the range boundary (b and d; $n = 14$ extant populations). Full results for all parameter combinations are provided in the online supplement.

\newpage

\large{\textbf{Figure 1}}
\begin{figure}
\centering
\includegraphics[width=1\textwidth]{"/Users/Topher/Desktop/RangeShifts/ShiftingSlopesOther/SchematicFigures/SimExample"}
\vspace{-5mm}
\label{fig:SimExample}
\end{figure}

\clearpage

\large{\textbf{Figure 2}}
\begin{figure}
\centering
\includegraphics[width=1\textwidth]{"/Users/Topher/Desktop/RangeShifts/ShiftingSlopesOther/ResultFigures/MainExtinction"}
\vspace{-5mm}
\label{fig:ExtProb}
\end{figure}

\clearpage

\large{\textbf{Figure 3}}
\begin{figure}
\centering
\includegraphics[width=1\textwidth]{"/Users/Topher/Desktop/RangeShifts/ShiftingSlopesOther/ResultFigures/DispComposite"}
\vspace{-5mm}
\label{fig:Disp}
\end{figure}

\end{document}