\documentclass[12pt, oneside]{article}
\usepackage{geometry}                		
\geometry{letterpaper}                   		
\usepackage{graphicx}
\usepackage{graphics}
\usepackage{hyperref}
\usepackage{fancybox}
\usepackage[centertags]{amsmath}
\usepackage{amssymb}
\usepackage{amsthm}
\usepackage{natbib}				
\usepackage{fullpage}
\usepackage{placeins}
\usepackage{setspace}
\usepackage{lineno}
\usepackage{color}
\usepackage{mathptmx}
\usepackage{authblk}

\DeclareRobustCommand{\firstsecond}[2]{#1}

\newcommand{\mb}{\mathbf}
\newcommand{\bs}{\boldsymbol}
\newcommand{\wt}{\widetilde}
\newcommand{\s}{^{(s)}}

\title{Spatial population structure determines extinction risk in climate-induced range shifts}

\date{}

\author[1,*]{Christopher Weiss-Lehman}
\author[1]{Allison K. Shaw}


\affil[1]{Ecology, Evolution, and Behavior, University of Minnesota}
\affil[*]{Denotes corresponding author}

\begin{document}
\maketitle

\begin{flushleft}
Corresponding author details: \\
Christopher Weiss-Lehman \\
email: cweissle@umn.edu \\
phone: +1 (720) 590 2278 \\
\end{flushleft}

\doublespacing
\linenumbers

\newpage

\section*{Abstract}
Climate change is an increasingly severe threat facing populations around the globe, necessitating a robust understanding of the ecological and evolutionary mechanisms dictating population responses. Population dynamics of range shifts, among the most commonly observed responses to climate change, can be influenced by many factors, including evolution of key traits, the degree of local adaptation, and the nature of the range edge. Here, we use an individual-based model to explore the interacting roles of these factors in the dynamics of climate-induced range shifts. We show that aspects of the spatial population structure within the initial range, in particular the potential for local adaptation, severely increased a population's extinction risk. Further, and contrary to expectations, we show that evolution of heightened dispersal during range shifts was unable to rescue faltering populations. Rather, a population's fate during climate change was determined by the composition of dispersal phenotypes that evolved within the initial range; only populations consisting of highly dispersive individuals prior to the onset of climate change survived. Our results demonstrate that dispersal evolution alone may be insufficient to save a range shifting population and that initial spatial population structure plays a pivotal role in determining the outcome of climate-induced range shifts.

\begin{flushleft}
\textbf{Key words:} range shifts, eco-evolutionary dynamics, local adaptation, individual-based model
\end{flushleft}

\newpage

\section*{Introduction}
Climate change is expected to dramatically reshape global biogeographic patterns as some species shift their ranges to track changing environmental conditions~\citep{gonzalez2010global}. These range shifts are generally predicted to proceed upwards in latitude, elevation, or both as average global temperatures continue to rise~\citep{loarie2009velocity}. Indeed, contemporary range shifts have already been observed in a wide variety of taxa, ranging from algae to mammals~\citep{chen2011rapid, parmesan2006ecological}. Such range shifts present significant challenges to current and future conservation efforts as they can result in the extinction of populations failing to track a changing climate~\citep{parmesan2006ecological} as well as the creation of novel species assemblages~\citep{hobbs2009novel}. Understanding the ecological and evolutionary dynamics of such climate-induced range shifts will play a key role in informing current and future conservation work.

Large-scale population movements have been studied for decades in the context of range expansions (e.g. of invasive or reintroduced species), leading to a robust understanding of both the ecological~\citep{hastings2005spatial} and evolutionary~\citep{shine2011evolutionary, excoffier2009genetic} mechanisms shaping such expansions. For example, while the speed of a range expansion can be well approximated by a combination of the species' intrinsic growth rate and dispersal ability~\citep{hastings2005spatial}, recent research demonstrates that evolution in both of these traits can increase both the mean and variance of expansion speed through time~\citep{weiss2017rapid, ochocki2017rapid, szHucs2017rapid, shaw2015dispersal, phillips2015evolutionary}. As fundamentally similar spatial processes, it is likely that range shifts will also be subject to these ecological and evolutionary mechanisms known to drive range expansions. However, range shifts involve several additional complications absent from range expansions, which must be considered when predicting the dynamics of a shifting population. In particular, range shifts occur in populations with far more complex spatial structure compared to most range expansions, which typically begin from the successful establishment and spread of a small, founding population~\citep{hastings2005spatial}. While these founding populations often lack any significant spatial structure, populations undergoing range shifts are characterized by a spatial population structure formed by aspects of the previously stable ranges. For example, population ranges can vary in their potential for local adaptation throughout the range, the nature of the range edge, and the spatial distribution of key traits. 

Each of these factors relating to spatial population structure has the potential to affect the dynamics of range shifts under changing climatic conditions. For example, the underlying mechanism responsible for the gradient in population size from the range core to the edge (i.e. declines in carrying capacity versus growth rate) alters a population's extinction risk during climate driven range shifts~\citep{henry2013eco}. Additionally, the potential for local adaptation throughout a range has been related to extinction risk during range shifts. Specifically, a low potential for local adaptation can decrease a population's ability to track a changing climate if dispersal occurs in a stepping stone manner, allowing some individuals to block the establishment of better adapted genotypes~\citep{atkins2010local}. While these aspects of spatial population structure have been shown to impact the dynamics of climate-induced range shifts in isolation, it is unclear how and if they might interact. 

Further, given the importance of rapid trait evolution in range expansions~\citep{weiss2017rapid, ochocki2017rapid, szHucs2017rapid, shaw2015dispersal, phillips2015evolutionary}, it is necessary to consider the interplay between aspects of spatial population structure and the role of rapid evolution during range shifts. In asexual species, for example, local adaptation has been shown to interact with dispersal evolution during climate change, driving increased dispersal probability as genotypes shift to keep pace with their environmental optimum~\citep{hargreaves2015fitness}. However, it is unclear how these two processes might interact in a sexually reproducing species in which dispersal and local adaptation are directly linked via gene flow. Under sexual reproduction, evolution of increased dispersal could simultaneously reduce local adaptation within a population due to increased gene flow throughout the range. In fact, long-distance pollen dispersal in flowering plants has been shown to restrict local adaptation and, when pollen dispersal sufficiently outpaces seed dispersal, to lead to ecological niche shifts, rather than spatial range shifts, in response to simulated climate change~\citep{aguilee2016pollen}. In addition to potential interactions between local adaptation and dispersal evolution, the nature of the range edge could influence the potential for rapid trait evolution during range shifts. For example, the severity of the environmental gradient forming the range edge has been shown to alter the spatial distribution of dispersal phenotypes throughout the range~\citep{henry2013eco, hargreaves2014evolution}, thus altering the diversity of dispersal genotypes present for subsequent evolution during range shifts.

Here, we assess the interaction of multiple aspects of spatial population structure with trait evolution in sexually reproducing populations undergoing climate-induced range shifts. We develop an individual-based model capable of incorporating a wide variety of spatial population structures in which males and females are defined by two genetically determined traits, thus allowing for both evolutionary and ecological responses to climate change. One trait determines dispersal ability while the second defines an individual's environmental niche. Using this model, we vary both the potential for local adaptation within the range and the nature of the range edge to ascertain how they interact with each other and with the process of trait evolution to impact a population's ability to track a changing climate. By contrasting the dynamics of extant and extinct populations, we isolate the factors most strongly contributing to extinction risk during climate change.

\section*{Methods}
A full description of the individual-based model using the Overview, Design concepts, and Details protocol~\citep{grimm2010odd} is available in Appendix A, while we present a brief summary here. Population dynamics occurred within discrete habitat patches embedded in a two dimensional lattice in which environmental conditions varied along the $x$ dimension but remained constant along the $y$ dimension (Fig. A1). Landscapes were unbounded in the $x$ dimension but defined by a fixed width in the $y$ dimension. Thus, the $x$ dimension defined the environmental context of the population and the $y$ dimension allowed for variation in population dynamics under identical environmental conditions. To simulate climate change, environmental conditions shifted at a constant rate along the $x$ dimension. Generations were non-overlapping and consisted of discrete dispersal and reproduction phases (Fig. A2).

Individuals were characterized by two traits (dispersal and an environmental niche), both defined by a set of $5$ quantitative diploid loci. While the number of loci was arbitrary, $5$ was chosen as a compromise between computational restrictions and the likely polygenic nature of such complex traits. The dispersal trait defined an individual's expected dispersal distance, assuming an exponential dispersal kernel. An individual's environmental niche value allowed for local adaptation; the closer the niche value to the environmental optimum of the individual's patch, the higher the individual's realized fitness. The environmental optimum of individual patches could then be systematically varied across the range to allow for different degrees of local adaptation (i.e. larger changes in environmental optima allowed for greater local adaptation of the population). In addition to the potential for local adaptation, simulated ranges were characterized by a decline in patch carrying capacity from the range center to the edge, the severity of which could be adjusted without altering the total carrying capacity of the landscape (see Appendix A). Reproduction within each patch occurred via a stochastic implementation of the classic Ricker model~\citep{ricker1954stock, melbourne2008extinction}. Parental pairs formed via random sampling of the local population (with replacement) weighted by individual fitness. Allele inheritance was subject to mutation and assumed no linkages among loci.

To determine the effect of spatial population structure on the eco-evolutionary dynamics of range shifts, we varied parameter combinations to explore the interacting roles of local adaptation and the severity of the gradient in environmentally suitable habitat at the range edge (Table A1 and A2). Specifically, we considered a factorial combination of three experimental factors: (1) no, low, and high potential for local adaptation, (2) shallow, moderate, and stark gradients in suitable habitat at the range edge, and (3) slow, moderate, and fast speeds of climate change. This yielded a total of $27$ different scenarios, each explored with $200$ simulations. Each simulation ran for $2150$ generations with stable climate conditions for the first $2000$, followed by $100$ generations of climate change and a final $50$ generations of stable conditions. Figure 1 shows an example of a single population responding to a moderate speed of climate change. For each scenario, we evaluated the role of dispersal evolution and initial spatial population structure in driving the dynamics of the range shifting populations. We primarily discuss simulations using a moderate speed of climate change in the main text, but present the results for slow and fast speeds of climate change in Appendix B.

We calculated dispersal evolution in each patch throughout the landscape as the change in mean dispersal phenotype from the beginning of the period of climate change to the end. For this analysis, we defined individual patches by their relative location within the range rather than with their fixed spatial coordinates (e.g. leading edge vs. core populations). Due to local extinctions, not all patches were occupied at the end of the period of climate change. To quantify dispersal evolution in these patches, we used data from the last generation in which the population had at least $10$ individuals. Changes in mean dispersal phenotype were calculated by subtracting the initial mean dispersal phenotype from the value at the end of climate change (or at the last generation of at least $10$ individuals occupying the patch in the case of population extinctions); positive values indicate an increase in the mean dispersal phenotype. All simulations and data processing were performed in R version $3.4.4$~\citep{team2000r} and the code is available at \url{https://github.com/tpweiss06/ShiftingSlopes}.

\section*{Results}
In all scenarios, some populations shifted their ranges in response to climate change. However the proportion of extinct populations that failed to track the changing climate depended on the initial spatial population structure. Populations defined by a higher potential for local adaptation and by stark habitat gradients at the range edge experienced the greatest probability of extinction due to climate change (quantified by the proportion of simulated populations to go extinct through time; Fig. 2). While both aspects of a population's range influenced extinction probabilities, the potential for local adaptation drove more dramatic changes to extinction risk, with greater changes in environmental optima across the landscape causing severe increases in the probability of extinction during climate change. We varied both parameters widely (the potential for local adaptation doubled from the low to high scenario and the parameter defining the severity of the environmental gradient was increased by a factor of 100 from shallow to stark gradients; Table A2), suggesting that potential for local adaptation may be the stronger driver of extinction risk during climate-induced range shifts across a wide region of parameter space and corresponding biological scenarios. Additionally, as expected, the pace of climate change also influenced extinction probabilities with faster climate change corresponding to greater extinction risk (Fig. B1 \& B2). However, this effect was independent of the roles of local adaptation and the habitat gradient at the range edge in determining the extinction probability during a range shift.

Counterintuitively, populations that survived climate change tended to be characterized by reduced fitness at the range margins prior to the onset of climate change compared to populations that went extinct (Fig. B3-B5). Essentially, populations with initially higher degrees of local adaptation at the range edges, and thus greater fitness, were more likely to go extinct during climate change. This pattern was most evident in the simulations with either (1) a gradual environmental gradient at the range edge or (2) a high potential for local adaptation. As expected, there was no spatial variation in fitness for populations with no potential for local adaptation. 

Dispersal evolution is predicted to play a key role in aiding populations as they shift to track a changing climate. While some, individual simulations confirmed these expectations with average dispersal phenotypes increasing through time (e.g. Fig. 1), examining all simulations from each experimental scenario revealed no differences in the magnitude or direction of dispersal evolution between successful and extinct populations (Fig. 3a\&b). Populations in all parameter combinations experienced both increases and decreases in average dispersal phenotypes, with all distributions of observed changes in dispersal phenotypes centered on $0$ (Fig. B3-B5). The similarity in evolved changes in dispersal between surviving and extinct populations suggests that dispersal evolution alone cannot explain which populations successfully tracked moving conditions and which became extinct.

Instead, the initial distribution of dispersal phenotypes prior to the onset of climate change played a key role in determining a population's fate. A range of dispersal phenotypes evolved in populations over the $2000$ generations of stable climatic conditions in response to the potential for local adaptation and the severity of the habitat gradient at the range edge (Fig. B9-B11). Populations that survived climate change were composed primarily of individuals with heightened dispersal phenotypes (Fig. 3c\&d). In fact, comparing the full distribution of initial dispersal phenotypes present in a given experimental scenario to the distribution of phenotypes from surviving populations revealed a threshold value delineating individuals from surviving versus extinct populations. Comparison of the different experimental scenarios revealed this threshold to be constant for a given speed of climate change (Fig. B9-B11). To explain this phenomenon, we used the well-known approximation for the speed of an expanding population, $2\sqrt{rD}$~\citep{hastings2005spatial}, in which $r$ is the intrinsic growth rate and $D$ is the diffusion coefficient, to calculate the dispersal phenotype necessary to produce an expansion wave exactly matching the speed of climate change used in our simulations (see the model description in Appendix A). This dispersal phenotype matched the observed threshold value distinguishing surviving from extinction populations in all experimental scenarios (Figures B9-B11, vertical dashed line). Thus, surviving populations in each scenario happened to be the lucky few already composed primarily of individuals with dispersal phenotypes capable of spreading at the pace of climate change, rather than populations in which heightened dispersal evolved over time in response to climate change.

\section*{Discussion}
Range shifts due to climate change represent a global threat to biodiversity and much recent research has focused on exploring the underlying ecological and evolutionary dynamics of such range shifts to inform conservation efforts. We developed an individual-based model to explore the eco-evolutionary dynamics of climate-induced range shifts in sexually reproducing, diploid populations with both dispersal and environmental niche traits defined by multiple loci. In contrast, previous models have focused on a subset of these factors: ecological dynamics (e.g.~\citep{brooker2007modelling}), evolution in a single trait only (e.g.~\citep{atkins2010local, henry2013eco}), and relatively simple genetic scenarios (e.g. single-locus haploid genetics in asexual populations~\citep{boeye2013more, hargreaves2015fitness}).  Here, we tested the generality of previous results and the interplay of eco-evolutionary dynamics under increased levels of biological complexity. Specifically, we demonstrated the role of spatial population structure, in the form of local adaptation and the environmental gradient defining the range edge, in determining extinction risk for range shifting populations via impacts on the initial distribution of dispersal phenotypes and environmental niche values.

Our results suggest that populations most likely to keep pace with climate change will be those with little to no local adaptation within the pre-expansion, stable range and in locations with shallow environmental gradients defining the range edge (Fig. 2). A survey of the scientific literature found evidence for local adaption in approximately $71\%$ of studies, suggesting a high prevalence of local adaptation in natural populations~\citep{hereford2009quantitative}. Further, a recent meta-analysis of $1400$ bird, mammal, fish, and tree species found no evidence for consistent declines in abundance towards range edges~\citep{dallas2017species}, suggesting many species exhibit similar abundances at the edge and center of their ranges similar to the stark environmental gradients imposed in our study. While some of these patterns could represent a publication bias, for example against negative results in studies of local adaptation, combined with our results they suggest many species will face elevated extinction risks in climate-induced range shifts due to the spatial population structure of their initial ranges.

Our results emphasize the importance of the initial distribution of dispersal phenotypes composing the stable range in determining a population's extinction risk during climate change (Fig. 3c\&d). Survival in the face of climate change was primarily determined by the dispersal phenotypes making up the population, specifically whether the population included individuals with dispersal phenotypes at or above a threshold value. Importantly, the threshold necessary to survive climate change itself was constant across all parameter combinations for a given speed of climate change (Fig. B9-B11). Scenarios with no potential for local adaptation and gradual environmental gradients had larger proportions of high dispersal phenotypes under stable climate conditions, and therefore a lower probability of extinction during climate change. A high potential for local adaptation, in contrast, selected against such high dispersal phenotypes due to dispersal's homogenizing effect on population genetic structure~\citep{lenormand2002gene}. Similarly, a more severe habitat gradient at the range edge increased the risk of dispersing beyond the boundary of suitable habitat, resulting in selection against heightened dispersal~\citep{shaw2014population}. Thus, attributes defining the spatial structure of the range altered the distribution of dispersal phenotypes under stable climate conditions, subsequently determining the extinction risk of populations during climate change. Importantly, dispersal evolution during climate change was unable to counter the influence of initial spatial population structure on extinction dynamics.

While high dispersal phenotypes prior to climate change increased the probability that populations tracked changing conditions, it had the additional effect of reducing average fitness at the range edges when populations had a moderate to high potential for local adaptation (Fig. B3-B5). In the model, populations at the range edges tended to have lower abundance than populations in the range core, increasing their susceptibility to gene flow from the core~\citep{lenormand2002gene}. Thus, in populations with high dispersal phenotypes prior to climate change, increased gene flow from the core likely reduced fitness at the range edge by preventing adaptation to local conditions. As a result, the populations most likely to survive climate change were, counterintuitively, also those characterized by lower fitness at the range edges prior to the onset of climate change. While not all populations are characterized by small populations at the range edges~\citep{dallas2017species}, our results suggest that populations exhibiting high levels of local adaptation within their stable range are likely to be at greater risk of extinction during periods of climate change.

Previous research has suggested that evolution of increased dispersal ability during climate change may be a key mechanism in rescuing populations that would otherwise be unable to keep pace with shifting environmental conditions~\citep{boeye2013more}. Our results suggest this is not always the case, and in fact may only be possible under certain, relatively narrow conditions. Previous models showing that dispersal evolution may rescue populations during climate change have typically used relatively simple genetic frameworks to model dispersal, including haploid genetics with a single-locus defining dispersal~\citep{boeye2013more, hargreaves2015fitness}. As dispersal evolution during range expansions and shifts occurs via the spatial sorting of alleles contributing to heightened dispersal at the range edge~\citep{shine2011evolutionary}, such simplified genetic frameworks may allow more efficient sorting of such alleles compared to situations with more complex genetic structure underlying the dispersal trait. The negligible role played by dispersal evolution in our model (Fig. 3a\&b) suggests that when such simplifying assumptions are relaxed, the potential for population rescue via evolution of heightened dispersal is greatly reduced, thus increasing the role of the initial spatial population structure within the range in determining a population's fate under climate change.

\subsection*{Conclusion}
As climate change continues to threaten populations, communities, and ecosystems~\citep{chen2011rapid, hobbs2009novel, gonzalez2010global}, it is increasingly important to understand population responses to changing environmental conditions. In particular, a deeper, process-based understanding of extinction risk in populations undergoing range shifts will, in turn, allow more focused conservation interventions. Our results suggest that the initial spatial population structure, as determined by local adaptation and the environmental gradient at the range edge, has the potential to dramatically alter the extinction probability faced by species responding to climate change. Further, in contrast to other studies assuming more simplified genetic structures, we find very little role for the evolution of heightened dispersal abilities in allowing a population to successfully track climate change. Future work should continue to examine the interplay between initial conditions in range shifts and the potential for evolutionary rescue. As climate change continues to accelerate~\citep{chen2017increasing}, it is imperative to not only identify those factors leading to increased extinction risk in range shifting populations, but also to develop meaningful conservation strategies to mitigate such risk.

\section*{Acknowledgements}
We thank Lauren Shoemaker, Lauren Sullivan, and members of the Theory Under Construction group at the University of Minnesota for providing thoughtful comments on the manuscript. CWL was supported by startup funds from the University of Minnesota (to AKS), and AKS was supported in part by funding from the National Science Foundation. We acknowledge the Minnesota Supercomputing Institute (MSI) at the University of Minnesota for providing resources that contributed to the results reported within this paper (http://www.msi.umn.edu).

\bibliographystyle{amnat}
\bibliography{main_bib}

\newpage

\section*{Figure legends}
\paragraph{Figure 1.} A single example of a simulation with a high potential for local adaptation and a moderate habitat gradient defining the range edge. Information on the (a) abundance, (b) dispersal, and (c) fitness of individuals in each patch is shown for time periods beginning with the last generation of stable climate conditions ($t = 0$) to $40$ generations after the start of climate change. Log transformed mean dispersal phenotypes (b) are shown for each patch. Average patch fitness (c) was calculated based on the mean environmental niche trait of local individuals and the environmental optima for each patch.

\paragraph{Figure 2.} The cumulative probability of extinction due to climate change in different experimental scenarios. Graphs show the proportion of simulated populations that went extinct through time for scenarios with (a) no, (b) low, and (c) high potential for local adaptation, and in environments characterized by a shallow (solid line), moderate (dashed line), or stark (dotted line) gradient at the range edge.

\paragraph{Figure 3.} Patterns in the evolution and the initial distribution of the dispersal trait, highlighting extant simulations. Evolution in dispersal (a and b) is shown as the change in the mean dispersal phenotype of each patch from the beginning of the period of climate change to the end. Positive values indicate an increase in average dispersal ability in the patch. Initial distributions of the dispersal trait (c and d) are shown as log transformed dispersal phenotypes of individuals in populations after $2000$ generations of stable climate conditions. In all panels, values associated with extant populations are shown in dark blue. Results are shown for populations with no potential for local adaptation and a gradual environmental gradient at the range boundary (a and c; $n = 155$ extant populations) and for populations with a high potential for local adaptation and a stark gradient at the range boundary (b and d; $n = 14$ extant populations). Full results for all parameter combinations are provided in Appendix B.

\newpage

\large{\textbf{Figure 1}}
\begin{figure}
\centering
\includegraphics[width=1\textwidth]{"/Users/Topher/Desktop/RangeShifts/ShiftingSlopesOther/SchematicFigures/SimExample"}
\vspace{-5mm}
\label{fig:SimExample}
\end{figure}

\clearpage

\large{\textbf{Figure 2}}
\begin{figure}
\centering
\includegraphics[width=1\textwidth]{"/Users/Topher/Desktop/RangeShifts/ShiftingSlopesOther/ResultFigures/MainExtinction"}
\vspace{-5mm}
\label{fig:ExtProb}
\end{figure}

\clearpage

\large{\textbf{Figure 3}}
\begin{figure}
\centering
\includegraphics[width=1\textwidth]{"/Users/Topher/Desktop/RangeShifts/ShiftingSlopesOther/ResultFigures/DispComposite"}
\vspace{-5mm}
\label{fig:Disp}
\end{figure}

\end{document}