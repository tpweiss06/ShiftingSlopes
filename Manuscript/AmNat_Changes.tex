\documentclass[11pt]{article}
%DIF LATEXDIFF DIFFERENCE FILE
%DIF DEL AmNat_InitialSubmission.tex   Thu Sep 20 14:37:23 2018
%DIF ADD AmNat_Resubmission.tex        Fri Jan 25 11:53:55 2019
\usepackage[sc]{mathpazo} %Like Palatino with extensive math support
\usepackage{fullpage}
\usepackage[authoryear,sectionbib,sort]{natbib}
\linespread{1.7}
\usepackage[utf8]{inputenc}
\usepackage{lineno}
\usepackage{titlesec}
\titleformat{\section}[block]{\Large\bfseries\filcenter}{\thesection}{1em}{}
\titleformat{\subsection}[block]{\Large\itshape\filcenter}{\thesubsection}{1em}{}
\titleformat{\subsubsection}[block]{\large\itshape}{\thesubsubsection}{1em}{}
\titleformat{\paragraph}[runin]{\itshape}{\theparagraph}{1em}{}[. ]\renewcommand{\refname}{Literature Cited}

\usepackage{mathptmx}
\usepackage{hyperref}
\usepackage{geometry}
\usepackage[centertags]{amsmath}
\usepackage{amssymb}
\usepackage{amsthm}
\usepackage{fancybox}
\usepackage{graphicx}
\usepackage{graphics}
\newcommand{\s}{^{(s)}}

%%%%%%%%%%%%%%%%%%%%%
% Line numbering
%%%%%%%%%%%%%%%%%%%%%
\usepackage{lineno}
% Please use line numbering with your initial submission and
% subsequent revisions. After acceptance, please turn line numbering
% off by adding percent signs to the lines %\usepackage{lineno} and
% to %\linenumbers{} and %\modulolinenumbers[3] below.

\title{Spatial population structure determines extinction risk in climate-induced range shifts}

% This version of the LaTeX template was last updated on
% January 11, 2018.

%%%%%%%%%%%%%%%%%%%%%
% Authorship
%%%%%%%%%%%%%%%%%%%%%
% Please remove authorship information while your paper is under review,
% unless you wish to waive your anonymity under double-blind review. You
% will need to add this information back in to your final files after
% acceptance.

%\author{Owen E. Cook$^{1,\ast}$ \\ 
%Generic H. Collaborator$^{2,\dag}$ \\ 
%Additional Q. Expert$^{3}$}

\date{}
%DIF PREAMBLE EXTENSION ADDED BY LATEXDIFF
%DIF UNDERLINE PREAMBLE %DIF PREAMBLE
\RequirePackage[normalem]{ulem} %DIF PREAMBLE
\RequirePackage{color}\definecolor{RED}{rgb}{1,0,0}\definecolor{BLUE}{rgb}{0,0,1} %DIF PREAMBLE
\providecommand{\DIFaddtex}[1]{{\protect\color{blue}\uwave{#1}}} %DIF PREAMBLE
\providecommand{\DIFdeltex}[1]{{\protect\color{red}\sout{#1}}}                      %DIF PREAMBLE
%DIF SAFE PREAMBLE %DIF PREAMBLE
\providecommand{\DIFaddbegin}{} %DIF PREAMBLE
\providecommand{\DIFaddend}{} %DIF PREAMBLE
\providecommand{\DIFdelbegin}{} %DIF PREAMBLE
\providecommand{\DIFdelend}{} %DIF PREAMBLE
%DIF FLOATSAFE PREAMBLE %DIF PREAMBLE
\providecommand{\DIFaddFL}[1]{\DIFadd{#1}} %DIF PREAMBLE
\providecommand{\DIFdelFL}[1]{\DIFdel{#1}} %DIF PREAMBLE
\providecommand{\DIFaddbeginFL}{} %DIF PREAMBLE
\providecommand{\DIFaddendFL}{} %DIF PREAMBLE
\providecommand{\DIFdelbeginFL}{} %DIF PREAMBLE
\providecommand{\DIFdelendFL}{} %DIF PREAMBLE
%DIF END PREAMBLE EXTENSION ADDED BY LATEXDIFF
%DIF PREAMBLE EXTENSION ADDED BY LATEXDIFF
%DIF HYPERREF PREAMBLE %DIF PREAMBLE
\providecommand{\DIFadd}[1]{\texorpdfstring{\DIFaddtex{#1}}{#1}} %DIF PREAMBLE
\providecommand{\DIFdel}[1]{\texorpdfstring{\DIFdeltex{#1}}{}} %DIF PREAMBLE
%DIF END PREAMBLE EXTENSION ADDED BY LATEXDIFF

\begin{document}

\maketitle

%\noindent{} 1. University of Chicago, Chicago, Illinois 60637;

%\noindent{} 2. University of Toronto, Toronto, Ontario M5S 1A5, Canada;

%\noindent{} 3. Middle Eastern Technical University, Çankaya, Ankara 06800, Turkey.

%\noindent{} $\ast$ Corresponding author; e-mail: amnat@uchicago.edu.

%\noindent{} $\dag$ Deceased.

\bigskip

\textit{Manuscript elements}: Figure~1, figure~2, figure~3, online appendices~A and B (including figures~\DIFdelbegin \DIFdel{A1\&A2}\DIFdelend \DIFaddbegin \DIFadd{A1-A3}\DIFaddend ,  tables~\DIFdelbegin \DIFdel{A1\&A2}\DIFdelend \DIFaddbegin \DIFadd{A1-A4}\DIFaddend , and figures~B1-B11). Figure~1 and figure~3 are to print in color.

\bigskip

\textit{Keywords}: range shifts, \DIFdelbegin \DIFdel{eco-evolutionary dynamics, local adaptation, }\DIFdelend \DIFaddbegin \DIFadd{extinction, rapid evolution, dispersal evolution, }\DIFaddend individual-based model

\bigskip

\textit{Manuscript type}: Article. %Or e-article, note, e-note, natural history miscellany, e-natural history miscellany, comment, reply, invited symposium, or countdown to 150.

\bigskip

\noindent{\footnotesize Prepared using the suggested \LaTeX{} template for \textit{Am.\ Nat.}}

\linenumbers{}
\modulolinenumbers[3]

\newpage{}

\section*{Abstract}
Climate change is an \DIFdelbegin \DIFdel{increasingly severe }\DIFdelend \DIFaddbegin \DIFadd{escalating }\DIFaddend threat facing populations around the globe, necessitating a robust understanding of the ecological and evolutionary mechanisms dictating population responses. \DIFdelbegin \DIFdel{Population dynamics of range shifts, among the most commonly observed responses }\DIFdelend \DIFaddbegin \DIFadd{However, populations do not respond }\DIFaddend to climate change \DIFdelbegin \DIFdel{, can be influenced by many factors, including evolution of key traits, the degree of local adaptation, and the nature of the range edge}\DIFdelend \DIFaddbegin \DIFadd{in isolation, but rather in the context of their existing ranges. In particular, spatial population structure within a range (e.g. trait clines, starkness of range edges, etc.) likely interacts with other ecological and evolutionary processes during climate-induced range shifts}\DIFaddend . Here, we use an individual-based model to explore the interacting roles of these factors in \DIFdelbegin \DIFdel{the dynamics of climate-induced range shifts }\DIFdelend \DIFaddbegin \DIFadd{range shifts dynamics}\DIFaddend . We show that \DIFdelbegin \DIFdel{aspects of }\DIFdelend the spatial population structure\DIFdelbegin \DIFdel{within the initial range}\DIFdelend , in particular the \DIFdelbegin \DIFdel{potential for local adaptation}\DIFdelend \DIFaddbegin \DIFadd{slope of the gradient in a trait optimum}\DIFaddend , severely increased a population's extinction risk. Further, and contrary to expectations, we show that evolution of heightened dispersal during range shifts was unable to rescue faltering populations. Rather, a population's fate during climate change was determined by the composition of dispersal phenotypes \DIFdelbegin \DIFdel{that evolved within the initial range}\DIFdelend \DIFaddbegin \DIFadd{defining the population at equilibrium (i.e. before the onset of rapid climate change)}\DIFaddend ; only populations consisting of highly dispersive individuals \DIFdelbegin \DIFdel{prior to the onset of climate change }\DIFdelend survived. Our results demonstrate that dispersal evolution alone may be insufficient to save a range shifting population and that \DIFdelbegin \DIFdel{initial }\DIFdelend spatial population structure \DIFdelbegin \DIFdel{plays a pivotal role in determining the outcome of climate-induced }\DIFdelend \DIFaddbegin \DIFadd{can substantially increase extinction risk in }\DIFaddend range shifts.

\newpage{}

\section*{Introduction}
Climate change is expected to dramatically reshape global biogeographic patterns as some species shift their ranges to track changing environmental conditions~\citep{gonzalez2010global}. These range shifts are generally predicted to proceed upwards in latitude, elevation, or both as average global temperatures continue to rise~\citep{loarie2009velocity}. Indeed, contemporary range shifts have already been observed in a wide variety of taxa, ranging from algae to mammals~\citep{chen2011rapid, parmesan2006ecological}. Such range shifts present significant challenges to current and future conservation efforts as they can result in the extinction of populations failing to track a changing climate~\citep{parmesan2006ecological} as well as the creation of novel species assemblages~\citep{hobbs2009novel}. Understanding the ecological and evolutionary dynamics of such climate-induced range shifts will play a key role in informing current and future conservation work.

Large-scale population movements have been studied for decades in the context of range expansions (e.g. of invasive or reintroduced species), leading to a robust understanding of both the ecological~\citep{hastings2005spatial} and evolutionary~\citep{shine2011evolutionary, excoffier2009genetic} mechanisms shaping such expansions. For example, while the speed of a range expansion can be well approximated by a combination of the species' intrinsic growth rate and dispersal ability~\DIFdelbegin \DIFdel{\mbox{%DIFAUXCMD
\citep{hastings2005spatial}}%DIFAUXCMD
}\DIFdelend \DIFaddbegin \DIFadd{\mbox{%DIFAUXCMD
\citep{fisher1937wave, hastings2005spatial}}%DIFAUXCMD
}\DIFaddend , recent research demonstrates that evolution in both of these traits can increase both the mean and variance of expansion speed through time~\citep{weiss2017rapid, ochocki2017rapid, szHucs2017rapid, shaw2015dispersal, phillips2015evolutionary}. As fundamentally similar spatial processes, it is likely that range shifts will also be subject to these ecological and evolutionary mechanisms known to drive range expansions. However, range shifts involve several additional complications absent from range expansions, which must be considered when predicting the dynamics of a shifting population. In particular, range shifts \DIFdelbegin \DIFdel{occur in populations }\DIFdelend \DIFaddbegin \DIFadd{begin }\DIFaddend with far more complex spatial \DIFaddbegin \DIFadd{population }\DIFaddend structure compared to most range expansions\DIFdelbegin \DIFdel{, which typically begin from the successful establishment and spread of a small, founding population ~\mbox{%DIFAUXCMD
\citep{hastings2005spatial}}%DIFAUXCMD
. While these founding populations often lack any significant spatial structure, populations undergoing range shifts are characterized by a }\DIFdelend \DIFaddbegin \DIFadd{. Here and throughout the paper, we use the phrase spatial population structure to refer broadly to the spatial distribution of individuals, and their associated genotypes and phenotypes, within a population. Thus, spatial population structure can encompass factors such as spatial clines in trait values and the starkness of abundance declines characterizing the range edge. Range expansions (e.g. of invasive or reintroduced species) typically start from small founding populations brought to a new area and lacking any initial spatial population structure. While such populations form }\DIFaddend spatial population structure \DIFdelbegin \DIFdel{formed by aspects of the previously stable ranges}\DIFdelend \DIFaddbegin \DIFadd{during the expansion process~\mbox{%DIFAUXCMD
\citep{weiss2017rapid, ochocki2017rapid}}%DIFAUXCMD
, established populations respond to climate change in the context of preexisting spatial population structure}\DIFaddend . For example, population ranges can vary in \DIFdelbegin \DIFdel{their potential for local adaptation throughout the range, the nature }\DIFdelend \DIFaddbegin \DIFadd{the presence and severity of gradients in trait optima, the starkness }\DIFaddend of the range edge, and\DIFaddbegin \DIFadd{, thus, }\DIFaddend the spatial distribution of key traits. 

Each of these factors relating to spatial population structure has the potential to affect the dynamics of range shifts under changing climatic conditions. For example, the underlying mechanism \DIFdelbegin \DIFdel{responsible for the gradient in population size from the range core to the }\DIFdelend \DIFaddbegin \DIFadd{causing population declines at the range }\DIFaddend edge (i.e. declines in carrying capacity versus growth rate) \DIFdelbegin \DIFdel{alters }\DIFdelend \DIFaddbegin \DIFadd{can alter }\DIFaddend a population's extinction risk during climate driven range shifts~\citep{henry2013eco}. \DIFaddbegin \DIFadd{Further, in ranges characterized by a gradient in a trait optimum, the starkness of the range edge can impact the ability of peripheral populations to adapt to the local optimum, with stark range edges leading to better adaptation in peripheral populations~\mbox{%DIFAUXCMD
\citep{garcia1997genetic}}%DIFAUXCMD
, though the importance of this for range shifts has not been investigated. }\DIFaddend Additionally, \DIFdelbegin \DIFdel{the potential for local adaptation throughout a range has been related to extinction risk during range shifts . Specifically, a low potential for local adaptation can decrease a population's ability to track a changing climate if }\DIFdelend \DIFaddbegin \DIFadd{the nature of dispersal can interact with adaptation to a gradient in a trait optimum to impact range shift dynamics. When }\DIFaddend dispersal occurs in a stepping stone manner, \DIFdelbegin \DIFdel{allowing some individuals to }\DIFdelend \DIFaddbegin \DIFadd{maladapted individuals (i.e. individuals whose phenotypes do not match the local optimum) can }\DIFaddend block the establishment of better adapted genotypes \DIFaddbegin \DIFadd{so long as the mismatch is not too severe}\DIFaddend ~\citep{atkins2010local}. While these aspects of spatial population structure have been shown to impact the dynamics of climate-induced range shifts in isolation, it is unclear how and if they might interact. 

Further, given the importance of rapid trait evolution in range expansions~\citep{weiss2017rapid, ochocki2017rapid, szHucs2017rapid, shaw2015dispersal, phillips2015evolutionary}, it is necessary to consider the interplay between aspects of spatial population structure and the role of rapid evolution during range shifts. In asexual species, for example, local adaptation \DIFaddbegin \DIFadd{to a gradient in a trait optimum }\DIFaddend has been shown to interact with dispersal evolution during climate change, driving increased dispersal probability as genotypes shift to keep pace with their environmental optimum~\citep{hargreaves2015fitness}. However, it is unclear how these two processes might interact in a sexually reproducing species in which dispersal and local adaptation are directly linked via gene flow. Under sexual reproduction, evolution of increased dispersal could simultaneously reduce local adaptation \DIFaddbegin \DIFadd{to a gradient in a trait optimum }\DIFaddend within a population due to increased gene flow throughout the range\DIFaddbegin \DIFadd{~\mbox{%DIFAUXCMD
\citep{garcia1997genetic, kirkpatrick1997evolution}}%DIFAUXCMD
}\DIFaddend . In fact, long-distance pollen dispersal in flowering plants has been shown to restrict local adaptation and, when pollen dispersal sufficiently outpaces seed dispersal, to lead to ecological niche shifts, rather than spatial range shifts, in response to simulated climate change~\citep{aguilee2016pollen}. In addition to potential interactions between local adaptation and dispersal evolution, the \DIFdelbegin \DIFdel{nature }\DIFdelend \DIFaddbegin \DIFadd{starkness }\DIFaddend of the range edge could influence the potential for rapid trait evolution during range shifts \DIFdelbegin \DIFdel{. For example, the severity of the environmental gradient forming the range edge has been shown to alter the }\DIFdelend \DIFaddbegin \DIFadd{by altering the }\DIFaddend spatial distribution of dispersal phenotypes throughout the range~\citep{henry2013eco, hargreaves2014evolution}, thus altering the diversity of dispersal genotypes present for subsequent evolution during range shifts.

Here, we assess the interaction of \DIFdelbegin \DIFdel{multiple aspects of }\DIFdelend \DIFaddbegin \DIFadd{two mechanisms responsible for }\DIFaddend spatial population structure with trait evolution in sexually reproducing populations undergoing \DIFdelbegin \DIFdel{climate-induced }\DIFdelend range shifts. We develop an individual-based model capable of \DIFdelbegin \DIFdel{incorporating }\DIFdelend \DIFaddbegin \DIFadd{producing }\DIFaddend a wide variety of spatial population structures in which males and females are defined by two genetically determined traits, thus allowing for both evolutionary and ecological responses to climate change. One trait determines dispersal ability while the second defines an individual's environmental niche. Using this model, we vary both the \DIFdelbegin \DIFdel{potential for local adaptation within the range and the nature }\DIFdelend \DIFaddbegin \DIFadd{severity of the gradient in the niche optimum and the starkness }\DIFaddend of the range edge to ascertain how they interact with each other and with the process of trait evolution to impact a population's ability to track a changing climate. \DIFdelbegin \DIFdel{By }\DIFdelend \DIFaddbegin \DIFadd{Additionally, by }\DIFaddend contrasting the dynamics of extant and extinct populations, we isolate the factors most strongly contributing to extinction risk during climate change.

\section*{Methods}
A full description of the individual-based model using the Overview, Design concepts, and Details protocol~\citep{grimm2010odd} is available in Appendix A, while we present a brief summary here. Population dynamics occurred within discrete habitat patches embedded in a two dimensional lattice in which environmental conditions varied along the $x$ dimension but remained constant along the $y$ dimension (Fig. A1). Landscapes were unbounded in the $x$ dimension but defined by a fixed width \DIFaddbegin \DIFadd{and wrapping boundaries }\DIFaddend in the $y$ dimension. \DIFaddbegin \DIFadd{The optimum environmental niche value changed linearly along the $x$ dimension, thus allowing for the intrinsic formation of stable range boundaries when the optimum changed rapidly enough~\mbox{%DIFAUXCMD
\citep{kirkpatrick1997evolution, alleaume2006geographical, polechova2015limits, polechova2018sky}}%DIFAUXCMD
. However, to examine the dynamics of ranges in which the niche optimum does not change rapidly enough to form stable range limits, we additionally imposed extrinsic range limits to prevent continuous adaptation and spread of the population~\mbox{%DIFAUXCMD
\citep{alleaume2006geographical, garcia1997genetic}}%DIFAUXCMD
. Specifically, we systematically altered the decline in patch carrying capacities from the range core to the edge~\mbox{%DIFAUXCMD
\citep{alleaume2006geographical, henry2013eco, bocedi2014rangeshifter, mustin2009dynamics} }%DIFAUXCMD
in such a way that we could directly manipulate the starkness of the decline. Previous research has shown that using a decline in intrinsic growth rate as opposed to carrying capacity may impact extinction risk but does not alter the patterns of dispersal evolution during climate change~\mbox{%DIFAUXCMD
\citep{henry2013eco}}%DIFAUXCMD
. To maintain generality, we do not assume a specific mechanism behind the decline in carrying capacity, but it could represent a variety of range limiting mechanisms such as physiological limits to adaptation, the effects of competition, or underlying resource distributions~\mbox{%DIFAUXCMD
\citep{sexton2009evolution, holt2005theoretical, case2000interspecific}}%DIFAUXCMD
. }\DIFaddend Thus, the $x$ dimension defined the environmental context of the population and the $y$ dimension allowed for variation in population dynamics under identical environmental conditions. To simulate climate change, \DIFdelbegin \DIFdel{environmental conditions }\DIFdelend \DIFaddbegin \DIFadd{the patch carrying capacities and the gradient in the niche optimum }\DIFaddend shifted at a constant rate along the $x$ dimension. Generations were non-overlapping and consisted of discrete dispersal and reproduction phases (Fig. A2).

Individuals were characterized by two traits (dispersal and an environmental niche), both defined by a set of $5$ quantitative diploid loci. While the number of loci was arbitrary, $5$ was chosen as a compromise between computational restrictions and the likely polygenic nature of such complex traits. The dispersal trait defined an individual's expected dispersal distance, assuming an exponential dispersal kernel. An individual's \DIFdelbegin \DIFdel{environmental niche value allowed for local adaptation; the closer the niche value to the environmental optimum of the }\DIFdelend \DIFaddbegin \DIFadd{realized dispersal distance was then drawn from the dispersal kernel and dispersal direction was random and unbiased. Dispersal occurred in continuous space from the center of an }\DIFaddend individual's \DIFdelbegin \DIFdel{patch, the higher the }\DIFdelend \DIFaddbegin \DIFadd{current patch and the }\DIFaddend individual's \DIFdelbegin \DIFdel{realized fitness. The environmental optimum of individual patches could then be systematically varied across the range to allow for different degrees of local adaptation (i. e. larger changes in environmental optima allowed for greater local adaptation of the population). In addition to the potential for local adaptation, simulated ranges were characterized by a decline in patch carrying capacity from the range center to the edge, the severity of which could be adjusted without altering the total carrying capacity of the landscape (see Appendix A)}\DIFdelend \DIFaddbegin \DIFadd{new patch was then determined by the mapping from continuous space to discrete patches (see Appendix A). An individual's environmental niche value determined its fitness according to the local niche optimum. The closer an individual's niche value to the local environmental optimum, the higher the individual's realized fitness}\DIFaddend . Reproduction within each patch occurred via a stochastic implementation of the classic Ricker model~\citep{ricker1954stock, melbourne2008extinction}\DIFaddbegin \DIFadd{, scaled by the mean fitness of the patch}\DIFaddend . Parental pairs formed via random sampling of the local population (with replacement) weighted by individual fitness \DIFdelbegin \DIFdel{. }\DIFdelend \DIFaddbegin \DIFadd{such that individuals with a close match of their niche value to the local optimum produced more offspring on average. Thus, the model used a mixture of  hard selection (realized population growth declines with maladaptation relative to the niche optimum) and soft selection (probability of producing offspring depends on fitness relative to other individuals) for the evolutionary dynamics~\mbox{%DIFAUXCMD
\citep{wallace1975hard}}%DIFAUXCMD
. }\DIFaddend Allele inheritance was subject to mutation and assumed no linkages among loci. \DIFaddbegin \DIFadd{The mutation process was designed such that mutational input per generation was independent of the number of loci (see Appendix A) and with parameters corresponding to previous estimates from the literature~\mbox{%DIFAUXCMD
\citep{gilbert2017local}}%DIFAUXCMD
.
}\DIFaddend 

\DIFdelbegin \DIFdel{To determine the effect of spatial population structure on the eco-evolutionary dynamics of range shifts, we varied parameter combinations }\DIFdelend \DIFaddbegin \DIFadd{We varied parameter values }\DIFaddend to explore the interacting roles of \DIFdelbegin \DIFdel{local adaptation and }\DIFdelend the severity of the gradient in \DIFdelbegin \DIFdel{environmentally suitable habitat at the }\DIFdelend \DIFaddbegin \DIFadd{the niche optimum and the starkness of the }\DIFaddend range edge (Table \DIFdelbegin \DIFdel{A1 and }\DIFdelend A2) \DIFaddbegin \DIFadd{in forming spatial population structure at equilibrium and driving the subsequent eco-evolutionary dynamics of range shifts}\DIFaddend . Specifically, we considered a factorial combination of three experimental factors: (1) \DIFdelbegin \DIFdel{no, low, and high potential for local adaptation}\DIFdelend \DIFaddbegin \DIFadd{a flat, shallow, and steep gradient in the niche optimum across space}\DIFaddend , (2) shallow, moderate, and stark \DIFdelbegin \DIFdel{gradients in suitable habitat }\DIFdelend \DIFaddbegin \DIFadd{declines in carrying capacity }\DIFaddend at the range edge, and (3) slow, moderate, and fast speeds of climate change. This yielded a total of $27$ different scenarios, each explored with $200$ simulations. Each simulation ran for $2150$ generations with stable climate conditions for the first $2000$ \DIFaddbegin \DIFadd{to reach a spatial equilibrium}\DIFaddend , followed by $100$ generations of climate change and a final $50$ generations of stable conditions. Figure 1 shows an example of a single population responding to a moderate speed of climate change. For each scenario, we evaluated the role of \DIFdelbegin \DIFdel{dispersal evolution and initial }\DIFdelend \DIFaddbegin \DIFadd{the equilibrium }\DIFaddend spatial population structure \DIFdelbegin \DIFdel{in driving }\DIFdelend \DIFaddbegin \DIFadd{and dispersal evolution in }\DIFaddend the dynamics of the range shifting populations. We primarily discuss simulations \DIFdelbegin \DIFdel{using }\DIFdelend \DIFaddbegin \DIFadd{assuming }\DIFaddend a moderate speed of climate change in the main text, but present the results for slow and fast speeds of climate change in Appendix B.

We calculated dispersal evolution in each patch throughout the landscape as the change in mean dispersal phenotype from the beginning of the period of climate change to the end. For this analysis, we defined individual patches by their relative location within the range rather than with their fixed spatial coordinates (e.g. leading edge vs. core populations). Due to local extinctions, not all patches were occupied at the end of the period of climate change. To quantify dispersal evolution in these patches, we used data from the last generation in which the population had at least $10$ individuals. Changes in mean dispersal phenotype were calculated by subtracting the initial mean dispersal phenotype from the value at the end of climate change (or at the last generation of at least $10$ individuals occupying the patch in the case of population extinctions); positive values indicate an increase in the mean dispersal phenotype. All simulations and data processing were performed in R version $3.4.4$~\citep{team2000r} and the code is available at (links are available from the journal office).

\section*{Results}
In all scenarios, some populations shifted their ranges in response to climate change. However the proportion of extinct populations that failed to track the changing climate depended on the \DIFdelbegin \DIFdel{initial spatial population structure}\DIFdelend \DIFaddbegin \DIFadd{spatial characteristics of the range}\DIFaddend . Populations defined by a \DIFdelbegin \DIFdel{higher potential for local adaptation }\DIFdelend \DIFaddbegin \DIFadd{steep gradient in the niche optimum }\DIFaddend and by stark \DIFdelbegin \DIFdel{habitat gradients }\DIFdelend \DIFaddbegin \DIFadd{declines in carrying capacity }\DIFaddend at the range edge experienced the greatest probability of extinction due to climate change (quantified by the proportion of simulated populations to go extinct through time; Fig. 2). While both aspects of a population's range influenced extinction probabilities, the \DIFdelbegin \DIFdel{potential for local adaptation }\DIFdelend \DIFaddbegin \DIFadd{gradient in the niche optimum }\DIFaddend drove more dramatic changes to extinction risk, with \DIFdelbegin \DIFdel{greater changes in environmental optima across the landscape }\DIFdelend \DIFaddbegin \DIFadd{steeper gradients }\DIFaddend causing severe increases in the probability of extinction during climate change. We varied both parameters widely (the \DIFdelbegin \DIFdel{potential for local adaptation }\DIFdelend \DIFaddbegin \DIFadd{slope of the gradient in the niche optimum }\DIFaddend doubled from the \DIFdelbegin \DIFdel{low to high }\DIFdelend \DIFaddbegin \DIFadd{shallow to steep }\DIFaddend scenario and the parameter defining the \DIFdelbegin \DIFdel{severity of the environmental gradient }\DIFdelend \DIFaddbegin \DIFadd{starkness of the range edge }\DIFaddend was increased by a factor of \DIFdelbegin \DIFdel{100 }\DIFdelend \DIFaddbegin \DIFadd{$100$ }\DIFaddend from shallow to stark \DIFdelbegin \DIFdel{gradients}\DIFdelend \DIFaddbegin \DIFadd{edges}\DIFaddend ; Table A2), suggesting that \DIFdelbegin \DIFdel{potential for local adaptation }\DIFdelend \DIFaddbegin \DIFadd{the gradient in the niche optimum }\DIFaddend may be the stronger driver of extinction risk during climate-induced range shifts across a wide region of parameter space and corresponding biological scenarios. Additionally, as expected, the pace of climate change also influenced extinction probabilities with faster climate change corresponding to greater extinction risk (Fig. B1 \& B2). However, this effect was independent of the \DIFdelbegin \DIFdel{roles of local adaptation and the habitat gradient at }\DIFdelend \DIFaddbegin \DIFadd{gradient in the niche optimum and the starkness of }\DIFaddend the range edge in determining \DIFdelbegin \DIFdel{the }\DIFdelend extinction probability during \DIFdelbegin \DIFdel{a range shift}\DIFdelend \DIFaddbegin \DIFadd{range shifts}\DIFaddend .

\DIFaddbegin \DIFadd{In accordance with previous results, populations at the range edge had lower fitness than central populations at equilibrium and this effect was amplified with more gradual range edges~\mbox{%DIFAUXCMD
\citep{garcia1997genetic}}%DIFAUXCMD
. }\DIFaddend Counterintuitively, populations that survived climate change tended to be characterized by \DIFdelbegin \DIFdel{reduced }\DIFdelend \DIFaddbegin \DIFadd{even greater reductions in }\DIFaddend fitness at the range \DIFdelbegin \DIFdel{margins prior to the onset of climate change }\DIFdelend \DIFaddbegin \DIFadd{edges at equilibrium }\DIFaddend compared to populations that went extinct (Fig. B3-B5). \DIFdelbegin \DIFdel{Essentially, populations with initially higher degrees of local adaptation at the range edges, and thus greater fitness, were more likely to go extinct during climate change. This }\DIFdelend \DIFaddbegin \DIFadd{While discernible in all simulations with a non-zero slope in the gradient of the niche optimum, this }\DIFaddend pattern was most evident in the \DIFdelbegin \DIFdel{simulations with either (1) a gradual environmental gradient at the range edgeor (2) a high potential for local adaptation}\DIFdelend \DIFaddbegin \DIFadd{scenarios defined by a gradual range edge}\DIFaddend . As expected, there was no spatial variation in fitness for populations with no \DIFdelbegin \DIFdel{potential for local adaptation. }\DIFdelend \DIFaddbegin \DIFadd{variation in the niche optimum across space. Despite the spatial variation in fitness in some scenarios, variance in relative fitness within a patch was relatively low (about $0.3$ across all scenarios). This implies that (1) the reduction in fitness at the edge caused by the spatial gradient in the niche optimum was relatively uniform across all individuals, and (2) as a result, there was relatively low variance in reproductive success in these populations, meaning that evolution at the edge was not driven by only a handful of higher fitness individuals.
}\DIFaddend 

Dispersal evolution is predicted to play a key role in aiding populations as they shift to track a changing climate. While some \DIFdelbegin \DIFdel{, }\DIFdelend individual simulations confirmed these expectations with average dispersal phenotypes increasing through time (e.g. Fig. 1), examining all simulations from each experimental scenario revealed no differences in the magnitude or direction of dispersal evolution between successful and extinct populations (Fig. 3a\&b). Populations in all parameter combinations experienced both increases and decreases in average dispersal phenotypes, with all distributions of observed changes in dispersal phenotypes centered on $0$ (Fig. B3-B5). \DIFaddbegin \DIFadd{Further, calculating the coefficient of variation (CV) in dispersal genotypes over the $y$ dimension, in which environmental conditions did not vary, revealed that edge populations had between $3$ and $4$ times higher CVs than core populations across all scenarios, indicating genetic diversity at the edge was not a limiting factor in dispersal evolution. }\DIFaddend The similarity in evolved changes in dispersal between surviving and extinct populations suggests that dispersal evolution alone cannot explain which populations successfully tracked moving conditions and which became extinct.

Instead, the \DIFdelbegin \DIFdel{initial }\DIFdelend distribution of dispersal phenotypes \DIFdelbegin \DIFdel{prior to the onset of climate change }\DIFdelend \DIFaddbegin \DIFadd{at equilibrium }\DIFaddend played a key role in determining a population's fate. A range of dispersal phenotypes evolved in populations over the $2000$ generations of stable climatic conditions in response to the \DIFdelbegin \DIFdel{potential for local adaptation and the severity of the habitat gradient at the }\DIFdelend \DIFaddbegin \DIFadd{gradient in the niche optimum and the starkness of the }\DIFaddend range edge (Fig. B9-B11). Populations that survived climate change were composed primarily of individuals with heightened dispersal phenotypes (Fig. 3c\&d). \DIFdelbegin \DIFdel{In fact, comparing }\DIFdelend \DIFaddbegin \DIFadd{Previous research has demonstrated that defining the range edges via a decline in the intrinsic growth rate (as opposed to carrying capacity as done here) resulted in less dispersive phenotypes at the range edge~\mbox{%DIFAUXCMD
\citep{henry2013eco}}%DIFAUXCMD
, meaning extinction risks would be even higher under such a scenario. Comparing }\DIFaddend the full distribution of \DIFdelbegin \DIFdel{initial }\DIFdelend \DIFaddbegin \DIFadd{equilibrium }\DIFaddend dispersal phenotypes present in a given experimental scenario to the distribution of phenotypes \DIFaddbegin \DIFadd{just }\DIFaddend from surviving populations revealed a threshold value delineating individuals from surviving versus extinct populations. Comparison of the different experimental scenarios revealed this threshold to be constant for a given speed of climate change (Fig. B9-B11). To explain this phenomenon, we used the well-known approximation for the speed of an expanding population, $2\sqrt{rD}$~\DIFdelbegin \DIFdel{\mbox{%DIFAUXCMD
\citep{hastings2005spatial}}%DIFAUXCMD
}\DIFdelend \DIFaddbegin \DIFadd{\mbox{%DIFAUXCMD
\citep{fisher1937wave, hastings2005spatial}}%DIFAUXCMD
}\DIFaddend , in which $r$ is the intrinsic growth rate and $D$ is the diffusion coefficient, to calculate the dispersal phenotype necessary to produce an expansion wave exactly matching the speed of climate change \DIFdelbegin \DIFdel{used }\DIFdelend in our simulations (see the model description in Appendix A). This \DIFaddbegin \DIFadd{estimated }\DIFaddend dispersal phenotype matched the observed threshold value distinguishing surviving from extinction populations in all experimental scenarios (Figures B9-B11, vertical dashed line). Thus, surviving populations in each scenario happened to be the lucky few already composed primarily of individuals with dispersal phenotypes capable of spreading at the pace of climate change, rather than populations in which heightened dispersal evolved over time in response to climate change. \DIFaddbegin \DIFadd{However, this threshold effect weakened slightly in scenarios with a slow speed of climate change indicating a potential role for dispersal evolution if the climate were to change at a slow enough rate (Figure B11).
}\DIFaddend 

\section*{Discussion}
Range shifts due to climate change represent a global threat to biodiversity and much recent research has focused on exploring the underlying ecological and evolutionary dynamics of such range shifts to inform conservation efforts. We developed an individual-based model to explore the eco-evolutionary dynamics of climate-induced range shifts in sexually reproducing, diploid populations with both dispersal and environmental niche traits defined by multiple loci. In contrast, previous models have focused on a subset of these factors: ecological dynamics (e.g.~\citep{brooker2007modelling}), evolution in a single trait only (e.g.~\citep{atkins2010local, henry2013eco}), and relatively simple genetic scenarios (e.g. single-locus haploid genetics in asexual populations~\citep{boeye2013more, hargreaves2015fitness}).  Here, we tested the generality of previous \DIFdelbegin \DIFdel{results }\DIFdelend \DIFaddbegin \DIFadd{predictions of an important role for dispersal evolution in range shifts~\mbox{%DIFAUXCMD
\citep{boeye2013more, henry2013eco} }%DIFAUXCMD
}\DIFaddend and the interplay of eco-evolutionary dynamics under increased levels of biological complexity. Specifically, we demonstrated the role of spatial population structure, \DIFdelbegin \DIFdel{in }\DIFdelend \DIFaddbegin \DIFadd{driven by a gradient in the niche optimum and the starkness of }\DIFaddend the \DIFdelbegin \DIFdel{form of local adaptation and the environmental gradient defining the }\DIFdelend range edge, in determining extinction risk for range shifting populations via impacts on the \DIFdelbegin \DIFdel{initial }\DIFdelend \DIFaddbegin \DIFadd{equilibrium }\DIFaddend distribution of dispersal phenotypes and environmental niche values.

Our results suggest that populations \DIFdelbegin \DIFdel{most likely to keep pace with climate change will be those with little to no local adaptation within the pre-expansion, stable range and in locations with shallow environmental gradients defining the range edge }\DIFdelend \DIFaddbegin \DIFadd{characterized by local adaptation to a spatially varying trait optimum and by stark range edges will be less able to track changing climatic conditions }\DIFaddend (Fig. 2). A survey of the scientific literature found evidence for local adaption in approximately $71\%$ of studies, suggesting a high prevalence of local adaptation in natural populations~\citep{hereford2009quantitative}. \DIFaddbegin \DIFadd{While it is difficult to exactly map the gradient in the niche optimum to empirical measures of local adaptation, the parameters defining the steepest gradient used here resulted in the intrinsic formation of stable range boundaries, as seen in previous theoretical studies~\mbox{%DIFAUXCMD
\citep{kirkpatrick1997evolution, alleaume2006geographical, polechova2015limits, polechova2018sky}}%DIFAUXCMD
, suggesting they provide reasonable approximations of empirical patterns. }\DIFaddend Further, a recent meta-analysis of $1400$ bird, mammal, fish, and tree species found no evidence for consistent declines in abundance towards range edges~\citep{dallas2017species}, suggesting many species exhibit similar abundances at the edge and center of their ranges similar to the \DIFdelbegin \DIFdel{stark environmental gradients }\DIFdelend \DIFaddbegin \DIFadd{starkest range edges }\DIFaddend imposed in our study. While some of these patterns could represent a publication bias, for example against negative results in studies of local adaptation, combined with our results they suggest many species will face elevated extinction risks in climate-induced range shifts due to \DIFdelbegin \DIFdel{the }\DIFdelend \DIFaddbegin \DIFadd{their }\DIFaddend spatial population structure\DIFdelbegin \DIFdel{of their initial ranges}\DIFdelend .

Our results emphasize the importance of the \DIFdelbegin \DIFdel{initial }\DIFdelend \DIFaddbegin \DIFadd{equilibrium }\DIFaddend distribution of dispersal phenotypes \DIFdelbegin \DIFdel{composing the stable range }\DIFdelend \DIFaddbegin \DIFadd{(i.e. prior to the onset of rapid climate change) }\DIFaddend in determining a population's extinction risk during climate change (Fig. 3c\&d). \DIFdelbegin \DIFdel{Survival in the face of climate change was primarily determined by the dispersal phenotypes making up the population, specifically whether the population included individuals with dispersal phenotypes at or above a threshold value}\DIFdelend \DIFaddbegin \DIFadd{Populations primarily composed of high dispersal phenotypes at equilibrium successfully tracked changing climate conditions, while populations of lower dispersal phenotypes lagged behind the changing conditions to eventually go extinct}\DIFaddend . Importantly, the threshold \DIFdelbegin \DIFdel{necessary to survive climate change itself }\DIFdelend \DIFaddbegin \DIFadd{dividing the equilibrium dispersal phenotypes of successful and extinct populations }\DIFaddend was constant across all \DIFdelbegin \DIFdel{parameter combinations }\DIFdelend \DIFaddbegin \DIFadd{scenarios }\DIFaddend for a given speed of climate change (Fig. \DIFdelbegin \DIFdel{B9-B11). Scenarios with no potential for local adaptation and gradual environmental gradients had larger proportions of high dispersal phenotypes under stable climate conditions, and therefore a lower probability of extinction during climate change. A high potential for local adaptation, in contrast, selected against such high dispersal phenotypes due to dispersal' s homogenizing effect on population genetic structure~\mbox{%DIFAUXCMD
\citep{lenormand2002gene}}%DIFAUXCMD
}\DIFdelend \DIFaddbegin \DIFadd{B9-11). Thus, the difference in survival probability among scenarios was driven by the effects of spatial population structure on the evolution of dispersal ability throughout the range. Scenarios with a steep gradient in the niche optimum selected for lower dispersal phenotypes at equilibrium due to the potential mismatch of dispersing individuals' niche phenotype and their new location~\mbox{%DIFAUXCMD
\citep{kirkpatrick1997evolution}}%DIFAUXCMD
}\DIFaddend . Similarly, a \DIFdelbegin \DIFdel{more severe habitat gradient at the range edge increased }\DIFdelend \DIFaddbegin \DIFadd{stark range edge selected for lower dispersal due to }\DIFaddend the risk of dispersing \DIFdelbegin \DIFdel{beyond the boundary of suitable habitat, resulting in selection against heightened dispersal}\DIFdelend \DIFaddbegin \DIFadd{into unsuitable habitat}\DIFaddend ~\citep{shaw2014population}. \DIFdelbegin \DIFdel{Thus, attributes defining the spatial structure of the range altered the distribution of dispersal phenotypes under stable climate conditions, subsequently determining the extinction risk of populations during climatechange}\DIFdelend \DIFaddbegin \DIFadd{Previous research has documented a similar reduction in dispersal phenotypes due to an explicit mortality cost of dispersal~\mbox{%DIFAUXCMD
\citep{kubisch2013predicting}}%DIFAUXCMD
, whereas the costs to dispersal in our model arise intrinsically as a result of the spatial population structure. Such a cost to dispersal results in lower dispersal phenotypes at equilibrium, hampering a population's ability to successfully track a changing climate}\DIFaddend . Importantly, dispersal evolution during climate change was unable to counter the influence of initial spatial population structure on extinction dynamics.

While high dispersal phenotypes \DIFdelbegin \DIFdel{prior to climate change }\DIFdelend \DIFaddbegin \DIFadd{at equilibrium }\DIFaddend increased the probability that populations tracked changing conditions, \DIFdelbegin \DIFdel{it }\DIFdelend \DIFaddbegin \DIFadd{they }\DIFaddend had the additional effect of reducing average fitness at the range edges when \DIFdelbegin \DIFdel{populations had a moderate to high potential for local adaptation }\DIFdelend \DIFaddbegin \DIFadd{the niche optimum varied across the range }\DIFaddend (Fig. B3-B5). In the model, \DIFdelbegin \DIFdel{populations at the range edges }\DIFdelend \DIFaddbegin \DIFadd{range edge populations }\DIFaddend tended to have lower abundance than \DIFdelbegin \DIFdel{populations in the range core }\DIFdelend \DIFaddbegin \DIFadd{core populations}\DIFaddend , increasing their susceptibility to gene flow from the core~\DIFdelbegin \DIFdel{\mbox{%DIFAUXCMD
\citep{lenormand2002gene}}%DIFAUXCMD
}\DIFdelend \DIFaddbegin \DIFadd{\mbox{%DIFAUXCMD
\citep{kirkpatrick1997evolution, garcia1997genetic}}%DIFAUXCMD
}\DIFaddend . Thus, in populations with high dispersal phenotypes \DIFdelbegin \DIFdel{prior to climate change}\DIFdelend \DIFaddbegin \DIFadd{at equilibrium}\DIFaddend , increased gene flow from the core \DIFdelbegin \DIFdel{likely }\DIFdelend reduced fitness at the range edge \DIFdelbegin \DIFdel{by preventing adaptation to local conditions}\DIFdelend \DIFaddbegin \DIFadd{via gene swamping~\mbox{%DIFAUXCMD
\citep{lenormand2002gene}}%DIFAUXCMD
}\DIFaddend . As a result, the populations most likely to survive climate change were, counterintuitively, also those characterized by lower fitness at the range edges \DIFdelbegin \DIFdel{prior to the onset of climate change}\DIFdelend \DIFaddbegin \DIFadd{at equilibrium}\DIFaddend . While not all populations are characterized by small populations at the range edges~\citep{dallas2017species}, our results suggest that \DIFdelbegin \DIFdel{populations exhibiting high levels of local adaptation within their stable range are likely to be at greater risk of extinction during periods of climate change}\DIFdelend \DIFaddbegin \DIFadd{high fitness in edge populations may be a warning sign of future difficulty in tracking climate change when the population is structured along a spatial gradient in a trait optimum. For the purposes of this investigation, we assumed the niche optima shifted spatially with climate change, as would be expected if the niche optima corresponded to local temperature or precipitation conditions~\mbox{%DIFAUXCMD
\citep{davis2001range}}%DIFAUXCMD
. However, in systems characterized by a gradient in the niche optimum defined by other factors (e.g. geography or biotic interactions) the gradient might remain stable or even shift in an opposing direction to climate change. Future research should investigate the impact of a gradient in the niche optimum unrelated to climate on extinction risk during range shifts}\DIFaddend .

Previous research has suggested that evolution of increased dispersal ability during climate change may be \DIFdelbegin \DIFdel{a key mechanism in }\DIFdelend \DIFaddbegin \DIFadd{capable of }\DIFaddend rescuing populations that would otherwise be unable to keep pace with shifting environmental conditions~\DIFdelbegin \DIFdel{\mbox{%DIFAUXCMD
\citep{boeye2013more}}%DIFAUXCMD
}\DIFdelend \DIFaddbegin \DIFadd{\mbox{%DIFAUXCMD
\citep{boeye2013more, henry2013eco}}%DIFAUXCMD
}\DIFaddend . Our results suggest this is not always the case, and in fact may only be possible under certain, relatively narrow conditions. Previous models showing that dispersal evolution may rescue populations during climate change have typically used relatively simple genetic frameworks to model dispersal, including haploid genetics with a single-locus defining dispersal~\citep{boeye2013more, hargreaves2015fitness}. As dispersal evolution during range expansions and shifts occurs via the spatial sorting of alleles contributing to heightened dispersal at the range edge~\citep{shine2011evolutionary}, such simplified genetic frameworks may allow more efficient sorting of such alleles compared to situations with more complex genetic structure underlying the dispersal trait. \DIFdelbegin \DIFdel{The negligible role played by dispersal evolution in our model (Fig. 3a\&b) suggests that when such simplifying assumptions are relaxed, the potential for population rescue via evolution of heightened dispersal is greatly reduced, thus increasing the role of the initial spatial population structure within the range in determining a population's fate under climate change}\DIFdelend \DIFaddbegin \DIFadd{Additionally, the interaction of mutation and genetic architecture in different models (e.g. few mutations of large effects or many mutations of small effects) undoubtedly plays a role in dispersal evolution during range shifts. Increasing mutation rate or effect size might have the equivalent effect of a slower speed of climate change in allowing dispersal evolution to play a greater role in range shift dynamics. Further, life history has been shown to impact the maintenance of genetic diversity, and hence evolutionary potential, with stage and age-structured populations shown to harbor greater diversity then populations with non-overlapping generations as modeled here~\mbox{%DIFAUXCMD
\citep{ellner1996environmental}}%DIFAUXCMD
. Populations defined by more complex life histories might, therefore, contain more genetic diversity in dispersal at equilibrium, making evolution of increased dispersal during range shifts more likely. Thus, further research is needed to understand how factors such as genetic architecture, mutational dynamics, and life history might interact to shape the potential for population rescue via dispersal evolution during range shifts}\DIFaddend .

\section*{Conclusion}
As climate change continues to threaten populations, communities, and ecosystems~\citep{chen2011rapid, hobbs2009novel, gonzalez2010global}, it is increasingly important to understand population responses to changing environmental conditions. In particular, a deeper, process-based understanding of extinction risk in populations undergoing range shifts will, in turn, allow more focused conservation interventions. Our results suggest that \DIFdelbegin \DIFdel{the initial }\DIFdelend spatial population structure, as determined by \DIFdelbegin \DIFdel{local adaptation and the environmental gradient at }\DIFdelend \DIFaddbegin \DIFadd{gradients in the niche optimum and the starkness of }\DIFaddend the range edge, has the potential to dramatically alter the extinction \DIFdelbegin \DIFdel{probability faced by }\DIFdelend \DIFaddbegin \DIFadd{risk for }\DIFaddend species responding to climate change. Further, in contrast to other studies assuming more simplified genetic structures, we find very little role for the evolution of heightened dispersal abilities in allowing a population to successfully track climate change. Future work should continue to examine the \DIFdelbegin \DIFdel{interplay between initial conditions in range shifts and }\DIFdelend \DIFaddbegin \DIFadd{circumstances determining }\DIFaddend the potential for \DIFdelbegin \DIFdel{evolutionary rescue }\DIFdelend \DIFaddbegin \DIFadd{rescue via dispersal evolution in range shifts}\DIFaddend . As climate change continues to accelerate~\citep{chen2017increasing}, it is imperative to \DIFdelbegin \DIFdel{not only }\DIFdelend identify those factors leading to increased extinction risk in range shifting populations, \DIFdelbegin \DIFdel{but also }\DIFdelend \DIFaddbegin \DIFadd{and use that knowledge }\DIFaddend to develop meaningful conservation strategies to mitigate such risk.

%%%%%%%%%%%%%%%%%%%%%
% Acknowledgments
%%%%%%%%%%%%%%%%%%%%%
% You may wish to remove the Acknowledgments section while your paper 
% is under review (unless you wish to waive your anonymity under
% double-blind review) if the Acknowledgments reveal your identity.
% If you remove this section, you will need to add it back in to your
% final files after acceptance.

%\section*{Acknowledgments}

\newpage{}

\section*{Appendix A: Full model description}

% Please reset counters for the appendix (thus normally figure A1, 
% figure A2, table A1, etc.).

% In certain cases, it may be appropriate to have a PRINT appendix in
% addition to (or instead of) an online appendix. In this case, please 
% name the print appendix Appendix A, and any subsequent appendixes (if 
% there are any) should be named Online Appendix B, Online Appendix C,
% etc.

% Counters for each appendix should match the letter of that appendix.
% For example, tables in Appendix C should be numbered table C1, table C2,
% etc. This applies to tables, equations, and figures.

% It's better not to use the \appendix command, because we have some
% formatting peculiarities that \appendix conflicts with.

\renewcommand{\theequation}{A\arabic{equation}}
% redefine the command that creates the equation number.
\renewcommand{\thetable}{A\arabic{table}}
\setcounter{equation}{0}  % reset counter 
\setcounter{figure}{0}
\setcounter{table}{0}

\section*{Model overview}
\subsection*{Purpose} 
This model \DIFdelbegin \DIFdel{tests }\DIFdelend \DIFaddbegin \DIFadd{tested }\DIFaddend an evolving population's ability to track a changing climate under a variety of conditions. Specifically, populations \DIFdelbegin \DIFdel{are }\DIFdelend \DIFaddbegin \DIFadd{were }\DIFaddend simulated under different combinations of (1) the \DIFdelbegin \DIFdel{starkness of the range boundary }\DIFdelend \DIFaddbegin \DIFadd{slope of a gradient in the niche optimum }\DIFaddend and (2) the \DIFdelbegin \DIFdel{potential for local adaptation}\DIFdelend \DIFaddbegin \DIFadd{starkness of the range edge}\DIFaddend . In all simulations, an individual's expected dispersal distance and environmental niche \DIFdelbegin \DIFdel{are }\DIFdelend \DIFaddbegin \DIFadd{were }\DIFaddend defined by an explicit set of quantitative diploid loci subject to mutation, thus allowing both traits to evolve over time. All simulations \DIFdelbegin \DIFdel{begin }\DIFdelend \DIFaddbegin \DIFadd{began }\DIFaddend with stable climate conditions for $2000$ generations to allow the populations to reach a spatial equilibrium before the onset of climate change. Climate change \DIFdelbegin \DIFdel{is }\DIFdelend \DIFaddbegin \DIFadd{was }\DIFaddend then modeled as a constant, directional shift in \DIFdelbegin \DIFdel{the location of environmentally suitable habitat }\DIFdelend \DIFaddbegin \DIFadd{environmental conditions }\DIFaddend (see \textit{Submodels} below). Finally, simulations \DIFdelbegin \DIFdel{end }\DIFdelend \DIFaddbegin \DIFadd{ended }\DIFaddend with another short period of climate stability to assess the population's ability to persist and recover after shifting its range.

\subsection*{State variables and scales} 
The model \DIFdelbegin \DIFdel{simulates }\DIFdelend \DIFaddbegin \DIFadd{simulated }\DIFaddend a population of males and females characterized by diploid loci for both their expected dispersal distance and environmental niche. Space \DIFdelbegin \DIFdel{is }\DIFdelend \DIFaddbegin \DIFadd{was }\DIFaddend modeled as a lattice of discrete patches overlaying a continuous Cartesian coordinate system. Landscapes \DIFdelbegin \DIFdel{are }\DIFdelend \DIFaddbegin \DIFadd{were }\DIFaddend two dimensional with a fixed width along the $y$ \DIFdelbegin \DIFdel{axis }\DIFdelend \DIFaddbegin \DIFadd{dimension }\DIFaddend and without bounds on the $x$ \DIFdelbegin \DIFdel{axis}\DIFdelend \DIFaddbegin \DIFadd{dimension}\DIFaddend . Environmental conditions \DIFdelbegin \DIFdel{vary }\DIFdelend \DIFaddbegin \DIFadd{varied }\DIFaddend along the $x$ dimension but \DIFdelbegin \DIFdel{remain }\DIFdelend \DIFaddbegin \DIFadd{remained }\DIFaddend constant within the $y$ dimension. To avoid edge effects due to the fixed width of the $y$ dimension, the model \DIFdelbegin \DIFdel{employs }\DIFdelend \DIFaddbegin \DIFadd{employed }\DIFaddend wrapping boundaries such that if an individual \DIFdelbegin \DIFdel{disperses }\DIFdelend \DIFaddbegin \DIFadd{dispersed }\DIFaddend out of the landscape on one side, it \DIFdelbegin \DIFdel{appears }\DIFdelend \DIFaddbegin \DIFadd{would appear }\DIFaddend at the opposite end of the same column of the landscape. Patches \DIFdelbegin \DIFdel{are }\DIFdelend \DIFaddbegin \DIFadd{were }\DIFaddend defined by the location of the patch center in $x$ and $y$ coordinates and a patch width parameter defining the relationship between continuous Cartesian space and the discrete patches used for population dynamics (\textit{Submodels}). 

The model \DIFdelbegin \DIFdel{implements }\DIFdelend \DIFaddbegin \DIFadd{implemented }\DIFaddend climate change by shifting the location of \DIFdelbegin \DIFdel{a population's available habitat }\DIFdelend \DIFaddbegin \DIFadd{patch carrying capacities and the gradient in the niche optimum }\DIFaddend along the $x$ dimension of the landscape. \DIFdelbegin \DIFdel{Available habitat (i.e. a population's potential range) is defined by a center location on the }\DIFdelend \DIFaddbegin \DIFadd{Patch carrying capacities were defined by the location of the range center along the }\DIFaddend $x$ dimension, the \DIFdelbegin \DIFdel{severity }\DIFdelend \DIFaddbegin \DIFadd{starkness }\DIFaddend of the decline \DIFdelbegin \DIFdel{in habitat quality at the edge}\DIFdelend \DIFaddbegin \DIFadd{characterizing the range edges}\DIFaddend , and the width of the \DIFdelbegin \DIFdel{available habitat }\DIFdelend \DIFaddbegin \DIFadd{range }\DIFaddend along the $x$ dimension (See Figure A1). \DIFdelbegin \DIFdel{Further, a gradient in environmental conditions is imposed throughout the landscape to allow for local adaptation via matching of an individual's environmental niche to the local environmental conditions (}\textit{\DIFdel{Submodels}}%DIFAUXCMD
\DIFdel{). The severity of this gradient can be altered to change the potential for local adaptation (e.g. a shallower gradient will result in more similar environmental conditions throughout the range and therefore reduce the potential for local adaptation)}\DIFdelend \DIFaddbegin \DIFadd{The gradient in the niche optimum was linear and shifted at the same speed and in the same direction as carrying capacities during climate change}\DIFaddend .

[Figure A1 goes here]

\subsection*{Process overview and scheduling} 
Time \DIFdelbegin \DIFdel{is }\DIFdelend \DIFaddbegin \DIFadd{was }\DIFaddend modeled in discrete intervals defining single generations of the population (Fig. A2). Within each generation, individuals first \DIFdelbegin \DIFdel{disperse }\DIFdelend \DIFaddbegin \DIFadd{dispersed }\DIFaddend from their natal patches according to their phenotypes. After dispersal, reproduction \DIFdelbegin \DIFdel{occurs according to }\DIFdelend \DIFaddbegin \DIFadd{occurred via }\DIFaddend a stochastic implementation of the classic Ricker model~\citep{ricker1954stock} taking into account the mean fitness of individuals within the patch. Reproduction \DIFdelbegin \DIFdel{occurs }\DIFdelend \DIFaddbegin \DIFadd{occurred }\DIFaddend via random sampling of the local population (with replacement) weighted by individual relative fitness such that individuals with high relative fitness (as determined by the match between their environmental niche and local conditions) \DIFdelbegin \DIFdel{are }\DIFdelend \DIFaddbegin \DIFadd{were }\DIFaddend likely to produce multiple offspring while individuals with low relative fitness \DIFdelbegin \DIFdel{may }\DIFdelend \DIFaddbegin \DIFadd{might }\DIFaddend not produce any. Individuals \DIFdelbegin \DIFdel{inherit }\DIFdelend \DIFaddbegin \DIFadd{inherited }\DIFaddend one allele from each parent at each loci, assuming independent segregation and a mutation process. After reproduction, all individuals in the current generation \DIFdelbegin \DIFdel{perish }\DIFdelend \DIFaddbegin \DIFadd{perished }\DIFaddend and the offspring \DIFdelbegin \DIFdel{begin }\DIFdelend \DIFaddbegin \DIFadd{began }\DIFaddend the next generation with dispersal, resulting in discrete, non-overlapping generations. 

[Figure A2 goes here]

\section*{Design concepts}
\subsection*{Emergence} 
Emergent phenomena in this model \DIFdelbegin \DIFdel{include the spatial equilibrium of population abundances and trait values within the stable range, the demographic dynamics of the shifting population during climate change, }\DIFdelend \DIFaddbegin \DIFadd{included the spatial distribution of population abundance, dispersal abilities, and relative fitness throughout the range. (i.e. the spatial population structure). Additionally, the population dynamics during the range shift, including the extinction process, }\DIFaddend and the evolutionary trajectories of \DIFdelbegin \DIFdel{both expected dispersal distances and environmental niche values during climate change}\DIFdelend \DIFaddbegin \DIFadd{the dispersal and niche traits were all emergent phenomena in this model}\DIFaddend .

\subsection*{Stochasticity} 
All biological processes in this model \DIFdelbegin \DIFdel{are }\DIFdelend \DIFaddbegin \DIFadd{were }\DIFaddend stochastic including realized population growth in each patch, dispersal distances of each individual, and inheritance of loci. Environmental parameters \DIFdelbegin \DIFdel{are }\DIFdelend \DIFaddbegin \DIFadd{were }\DIFaddend fixed, however, and the process of climate change (i.e. the movement of \DIFdelbegin \DIFdel{environmentally suitable habitat }\DIFdelend \DIFaddbegin \DIFadd{patch carrying capacity }\DIFaddend through time) \DIFdelbegin \DIFdel{is }\DIFdelend \DIFaddbegin \DIFadd{was }\DIFaddend deterministic. Thus, the model \DIFdelbegin \DIFdel{removes }\DIFdelend \DIFaddbegin \DIFadd{removed }\DIFaddend the confounding influence of environmental stochasticity to focus on demographic and evolutionary dynamics of range shifts.

\subsection*{Interactions} 
Individuals in the model \DIFdelbegin \DIFdel{interact }\DIFdelend \DIFaddbegin \DIFadd{interacted }\DIFaddend via mating and density-dependent competition within patches. Additionally, the evolutionary trajectories of the two different traits \DIFdelbegin \DIFdel{have }\DIFdelend \DIFaddbegin \DIFadd{had }\DIFaddend the potential to interact via the relationship between gene flow \DIFaddbegin \DIFadd{(dispersal trait) }\DIFaddend and local adaptation \DIFaddbegin \DIFadd{(niche trait)}\DIFaddend . Further, \DIFdelbegin \DIFdel{aspects determining the spatial population structure of a population's range (potential for local adaptation and the nature }\DIFdelend \DIFaddbegin \DIFadd{the gradient in the niche optimum and the starkness }\DIFaddend of the range edge \DIFdelbegin \DIFdel{) can }\DIFdelend \DIFaddbegin \DIFadd{could }\DIFaddend interact with trait evolution within the range both during stable climate conditions and during climate change.

\subsection*{Desired output} 
After each model run, full details of all surviving individuals at the last time point \DIFdelbegin \DIFdel{are }\DIFdelend \DIFaddbegin \DIFadd{were }\DIFaddend recorded (spatial coordinates and loci values for both traits). If a population went extinct during the model run, the time of extinction \DIFdelbegin \DIFdel{is }\DIFdelend \DIFaddbegin \DIFadd{was }\DIFaddend recorded. For each occupied patch throughout the simulation, we aggregated data on population size, the dispersal trait, and \DIFdelbegin \DIFdel{local adaptation to environmental }\DIFdelend \DIFaddbegin \DIFadd{adaptation to local }\DIFaddend conditions. 

\section*{Details}
\subsection*{Initialization} 
The following parameters \DIFdelbegin \DIFdel{are }\DIFdelend \DIFaddbegin \DIFadd{were }\DIFaddend set at the beginning of each simulation and \DIFdelbegin \DIFdel{form }\DIFdelend \DIFaddbegin \DIFadd{formed }\DIFaddend the initial conditions of the model: the mean and variance for allele values of each trait, population size, location of \DIFdelbegin \DIFdel{environmentally suitable habitat}\DIFdelend \DIFaddbegin \DIFadd{the range center}\DIFaddend , number of generations for the pre-, post-, and \DIFdelbegin \DIFdel{during }\DIFdelend \DIFaddbegin \DIFadd{rapid }\DIFaddend climate change periods of the simulation, and all other necessary parameters for the submodels defined below. Simulated populations \DIFdelbegin \DIFdel{are }\DIFdelend \DIFaddbegin \DIFadd{were }\DIFaddend initialized in the center of the range and allowed to spread and equilibrate throughout the range during the period of stable climate conditions. This \DIFdelbegin \DIFdel{ensures }\DIFdelend \DIFaddbegin \DIFadd{ensured }\DIFaddend that the populations reacting to a changing climate truly \DIFdelbegin \DIFdel{represent }\DIFdelend \DIFaddbegin \DIFadd{represented }\DIFaddend the expected spatial distribution for a given range, rather than the initial parameter values used in the simulation \DIFdelbegin \DIFdel{. See Tables }\DIFdelend \DIFaddbegin \DIFadd{(Table }\DIFaddend A1\DIFdelbegin \DIFdel{\& A2 for a full list of parameter values used in the simulations described here}\DIFdelend \DIFaddbegin \DIFadd{). Initial population size was chosen to minimize the risk of stochastic extinction in the early stages of the simulation. The time frames defining climate change were designed to give a reasonable period for the population to reach a spatial equilibrium and a long enough period of climate change for extinction dynamics to play out. The number of patches defining the $y$ dimension and the relationship between Cartesian space and discrete patches were chosen to allow a reasonable number of patches to contribute to the eco-evolutionary dynamics of range shifts while not proving computationally restrictive}\DIFaddend .

\begin{table}
\renewcommand{\arraystretch}{1.5}
  \DIFdelbeginFL %DIFDELCMD < \begin{tabular}{ p{2cm} | p{8cm} | p{2cm} }
%DIFDELCMD <     %%%
\DIFdelendFL \DIFaddbeginFL \begin{tabular}{ p{2cm} | p{8cm} | p{4cm} }
    \DIFaddendFL \hline
    Parameter & Description & Value \\ \hline \hline
    $N_{1}$ & Initial population size (seeded across multiple patches) when beginning the simulations & $2500$ \DIFaddbeginFL \DIFaddFL{individuals }\DIFaddendFL \\
    $\beta_{1}$ & Center of \DIFdelbeginFL \DIFdelFL{environmentally suitable habitat }\DIFdelendFL \DIFaddbeginFL \DIFaddFL{the range }\DIFaddendFL before climate change & $0$ \\
    \DIFdelbeginFL \DIFdelFL{$\eta$ }%DIFDELCMD < & %%%
\DIFdelFL{Spatial dimensions of habitat patches in continuous space }%DIFDELCMD < & %%%
\DIFdelFL{$50$ }%DIFDELCMD < \\
%DIFDELCMD <     %%%
\DIFdelFL{$y_{max}$ }%DIFDELCMD < & %%%
\DIFdelFL{Number of patches the discrete lattice extends in the $y$ direction }%DIFDELCMD < & %%%
\DIFdelFL{$10$ }%DIFDELCMD < \\
%DIFDELCMD <     %%%
\DIFdelendFL $\hat{t}$ & Last \DIFdelbeginFL \DIFdelFL{time point }\DIFdelendFL \DIFaddbeginFL \DIFaddFL{generation }\DIFaddendFL of stable climate conditions & $2000$ \\
    $t_{\Delta}$ & Duration of climate change & $100$ \DIFaddbeginFL \DIFaddFL{generations }\DIFaddendFL \\
    $t_{max}$ & Total number of \DIFdelbeginFL \DIFdelFL{time points }\DIFdelendFL \DIFaddbeginFL \DIFaddFL{generations }\DIFaddendFL in the simulation & $2150$ \DIFaddbeginFL \DIFaddFL{generations }\DIFaddendFL \\
    \DIFdelbeginFL \DIFdelFL{$R$ }\DIFdelendFL \DIFaddbeginFL \DIFaddFL{$\eta$ }\DIFaddendFL & \DIFdelbeginFL \DIFdelFL{Intrinsic growth rate of the population }\DIFdelendFL \DIFaddbeginFL \DIFaddFL{Width of square habitat patches in Cartesian space }\DIFaddendFL & \DIFdelbeginFL \DIFdelFL{$2$ }\DIFdelendFL \DIFaddbeginFL \DIFaddFL{$50$ }\DIFaddendFL \\
    \DIFdelbeginFL \DIFdelFL{$K_{max}$ }\DIFdelendFL \DIFaddbeginFL \DIFaddFL{$y_{max}$ }\DIFaddendFL & \DIFdelbeginFL \DIFdelFL{Maximum achievable carrying capacity in the range }%DIFDELCMD < & %%%
\DIFdelFL{$100$ }%DIFDELCMD < \\ 
%DIFDELCMD <     %%%
\DIFdelFL{$\psi$ }%DIFDELCMD < & %%%
\DIFdelFL{Expected sex ratio in the population }%DIFDELCMD < & %%%
\DIFdelFL{$0.5$ }%DIFDELCMD < \\
%DIFDELCMD <     %%%
\DIFdelFL{$\hat{d}$ }%DIFDELCMD < & %%%
\DIFdelFL{Maximum achievable dispersal phenotype }%DIFDELCMD < & %%%
\DIFdelFL{$1000$ }%DIFDELCMD < \\
%DIFDELCMD <     %%%
\DIFdelFL{$\rho$ }%DIFDELCMD < & %%%
\DIFdelFL{Determines the slope of }\DIFdelendFL \DIFaddbeginFL \DIFaddFL{Number of patches the discrete lattice extends in }\DIFaddendFL the \DIFdelbeginFL \DIFdelFL{transition in dispersal phenotypes from $0$ to $D$ }\DIFdelendFL \DIFaddbeginFL \DIFaddFL{$y$ dimension }\DIFaddendFL & \DIFdelbeginFL \DIFdelFL{$0.5$ }\DIFdelendFL \DIFaddbeginFL \DIFaddFL{$10$ patches }\DIFaddendFL \\
    \DIFdelbeginFL \DIFdelFL{$\omega$ }%DIFDELCMD < & %%%
\DIFdelFL{Defines the strength of stabilizing selection on fitness traits }%DIFDELCMD < & %%%
\DIFdelFL{$3$ }%DIFDELCMD < \\
%DIFDELCMD <     %%%
\DIFdelFL{$U^{T}$ }%DIFDELCMD < & %%%
\DIFdelFL{Diploid mutation rate for each trait }%DIFDELCMD < & %%%
\DIFdelFL{$0.02$ for each trait }%DIFDELCMD < \\
%DIFDELCMD <     %%%
\DIFdelFL{$V_{m}^{T}$ }%DIFDELCMD < & %%%
\DIFdelFL{Mutational variance for each trait }%DIFDELCMD < & %%%
\DIFdelFL{$0.0004$ for each trait }%DIFDELCMD < \\
%DIFDELCMD <     %%%
\DIFdelFL{$L^{T}$ }%DIFDELCMD < & %%%
\DIFdelFL{Number of diploid loci defining each trait }%DIFDELCMD < & %%%
\DIFdelFL{$5$ for each trait }%DIFDELCMD < \\
%DIFDELCMD <     %%%
\DIFdelFL{$\mu_{1}^{f}$ }%DIFDELCMD < & %%%
\DIFdelFL{Initial mean allele value for the environmental niche trait }%DIFDELCMD < & %%%
\DIFdelFL{$0$ }%DIFDELCMD < \\
%DIFDELCMD <     %%%
\DIFdelFL{$\mu_{1}^{d}$ }%DIFDELCMD < & %%%
\DIFdelFL{Initial mean allele value for the dispersal trait }%DIFDELCMD < & %%%
\DIFdelFL{$-1$ }%DIFDELCMD < \\
%DIFDELCMD <     %%%
\DIFdelFL{$\sigma_{1}^{f}$ }%DIFDELCMD < & %%%
\DIFdelFL{Initial standard deviation of allele values for the environmental niche trait }%DIFDELCMD < & %%%
\DIFdelFL{$0.025$ }%DIFDELCMD < \\
%DIFDELCMD <     %%%
\DIFdelFL{$\sigma_{1}^{d}$ }%DIFDELCMD < & %%%
\DIFdelFL{Initial standard deviation of allele values for the dispersal trait }%DIFDELCMD < & %%%
\DIFdelFL{$1$ }%DIFDELCMD < \\
%DIFDELCMD <     %%%
\DIFdelendFL \hline
  \end{tabular}
\caption[LoF entry]{\DIFdelbeginFL \DIFdelFL{Simulation }\DIFdelendFL \DIFaddbeginFL \DIFaddFL{Values and descriptions for }\DIFaddendFL parameters \DIFdelbeginFL \DIFdelFL{held constant across all scenarios}\DIFdelendFL \DIFaddbeginFL \DIFaddFL{determining the initial conditions of simulations, the timing of climate change, and the relationship between Cartesian space and the lattice of discrete habitat patches}\DIFaddendFL .}
\DIFdelbeginFL %DIFDELCMD < %DIFDELCMD < \label{table:ConstPars}%%%
%DIFDELCMD < %%%
\DIFdelendFL \DIFaddbeginFL \label{table:InitPars}
\DIFaddendFL \end{table}

\subsection*{Submodels}
\paragraph{\DIFdelbegin \DIFdel{Environmentally suitable habitat}\DIFdelend \DIFaddbegin \DIFadd{Patch carrying capacities}\DIFaddend }
\DIFdelbegin \DIFdel{Environmentally suitable habitat is determined by the population's carrying capacity as it ranges in space }\DIFdelend \DIFaddbegin \DIFadd{Patch carrying capacity }\DIFaddend ($K_{x}$) \DIFdelbegin \DIFdel{. The carrying capacity is maximized in the center of the species' range ($K_{max}$) and declines with increasing }\DIFdelend \DIFaddbegin \DIFadd{varied along the $x$ dimension of the landscape, attaining its highest value at the range center and declining with }\DIFaddend distance from the center. Specifically, the carrying capacity at a location $x$ \DIFdelbegin \DIFdel{is }\DIFdelend \DIFaddbegin \DIFadd{was }\DIFaddend defined as the product of \DIFaddbegin \DIFadd{the maximum potential carrying capacity (}\DIFaddend $K_{max}$\DIFaddbegin \DIFadd{) }\DIFaddend and a function $f(x,t)$, where $f(x,t)$ \DIFdelbegin \DIFdel{ranges from }\DIFdelend \DIFaddbegin \DIFadd{was bounded between }\DIFaddend $1$ \DIFdelbegin \DIFdel{in }\DIFdelend \DIFaddbegin \DIFadd{and $0$ with its highest value corresponding to }\DIFaddend the range center\DIFdelbegin \DIFdel{to $0$ far away from the center and is defined as
follows 
}\DIFdelend \DIFaddbegin \DIFadd{. $f(x,t)$ was defined as
}\DIFaddend \begin{equation}
f(x,t)=
\begin{cases}
	\frac{e^{\gamma(x-\beta_{t}+\tau)}}{1+e^{\gamma(x-\beta_{t}+\tau)}} & x \leq \beta_{t} \\
	\frac{e^{-\gamma(x-\beta_{t}-\tau)}}{1+e^{-\gamma(x-\beta_{t}-\tau)}} & x > \beta_{t}
\end{cases}
\end{equation}
where $\beta_{t}$ \DIFdelbegin \DIFdel{defines }\DIFdelend \DIFaddbegin \DIFadd{defined }\DIFaddend the center of the \DIFdelbegin \DIFdel{area of suitable habitat }\DIFdelend \DIFaddbegin \DIFadd{range }\DIFaddend at time $t$, $\tau$ \DIFdelbegin \DIFdel{sets }\DIFdelend \DIFaddbegin \DIFadd{affected }\DIFaddend the width of the range, and $\gamma$ \DIFdelbegin \DIFdel{affects }\DIFdelend \DIFaddbegin \DIFadd{affected }\DIFaddend the slope of the function at the range \DIFdelbegin \DIFdel{boundaries }\DIFdelend \DIFaddbegin \DIFadd{edges }\DIFaddend (See Figure A1). \DIFdelbegin \DIFdel{To understand the relationship between $\gamma$ and the slope of $f(x,t)$ at the range boundary, the partial derivative of $f(x,t)$ over the $x$ dimension can be shown to be
}\begin{displaymath}
\DIFdel{f(x,t)=
\begin{cases}
	\frac{\gamma e^{\gamma(x-\beta_{t}+\tau)}}{(1+e^{\gamma(x-\beta_{t}+\tau})^{2}} & x \leq \beta_{t} \\
	\frac{-\gamma e^{-\gamma(x-\beta_{t}-\tau)}}{(1+e^{-\gamma(x-\beta_{t}-\tau})^{2}} & x > \beta_{t}
\end{cases}	
}\end{displaymath}
%DIFAUXCMD
\DIFdel{yielding a value of $\pm\frac{\gamma}{4}$ at the inflection points on either side of the range center ($x=\beta_{t}\pm\tau$).
}%DIFDELCMD < 

%DIFDELCMD < %%%
\DIFdel{Population dynamics occur }\DIFdelend \DIFaddbegin \DIFadd{Population dynamics occurred }\DIFaddend within discrete patches, so to calculate a $K_{x}$ value for a discrete patch from the continuous function $f(x,t)$, we \DIFdelbegin \DIFdel{use }\DIFdelend \DIFaddbegin \DIFadd{used }\DIFaddend another parameter defining the spatial scale of each patch ($\eta$\DIFdelbegin \DIFdel{; See Figure A1}\DIFdelend ). The local carrying capacity of a patch centered on $x$ ($K_{x}$) \DIFdelbegin \DIFdel{is }\DIFdelend \DIFaddbegin \DIFadd{was }\DIFaddend then calculated as the mean of $f(x,t)$ over the interval of the patch multiplied by $K_{max}$.
\begin{equation}
K_{x} = \frac{K_{max}}{\eta}\int_{x-\frac{\eta}{2}}^{x+\frac{\eta}{2}}f(x,t)dx
\end{equation}

\DIFdelbegin \DIFdel{By varying the parameters defining $f(x,t)$, we can change both the total carrying capacity of the population, summed across all patches throughout the range, (by altering both $\tau$ and }\DIFdelend \DIFaddbegin \DIFadd{To understand the relationship between }\DIFaddend $\gamma$ \DIFdelbegin \DIFdel{) }\DIFdelend and the slope \DIFdelbegin \DIFdel{at which $K_{x}$ declines to $0$ (by altering }\DIFdelend \DIFaddbegin \DIFadd{of $f(x,t)$ at the range edge, we calculated the partial derivative of $f(x,t)$ over the $x$ dimension as
}\begin{equation}
\DIFadd{\frac{\partial f(x,t)}{\partial x}=
\begin{cases}
	\frac{\gamma e^{\gamma(x-\beta_{t}+\tau)}}{(1+e^{\gamma(x-\beta_{t}+\tau})^{2}} & x \leq \beta_{t} \\
	\frac{-\gamma e^{-\gamma(x-\beta_{t}-\tau)}}{(1+e^{-\gamma(x-\beta_{t}-\tau})^{2}} & x > \beta_{t}
\end{cases}	
}\end{equation}
\DIFadd{yielding a value of $\pm\frac{\gamma}{4}$ at the inflection points on either side of the range center ($x=\beta_{t}\pm\tau$). Thus, altering }\DIFaddend $\gamma$ \DIFdelbegin \DIFdel{). Changing the slope affects not only the rate at which $K_{x}$ declines at the range boundaries (our focus), but it also alters the total carrying capacity of the population. To avoid this confounding factor, we fix }\DIFdelend \DIFaddbegin \DIFadd{directly altered the starkness of the range edge. However, changing $\gamma$ also changed }\DIFaddend the total area under \DIFdelbegin \DIFdel{the curve }\DIFdelend $f(x,t)$ \DIFdelbegin \DIFdel{. The }\DIFdelend \DIFaddbegin \DIFadd{as can be seen in the }\DIFaddend indefinite integral of $f(x,t)$\DIFdelbegin \DIFdel{can be shown to be
}\DIFdelend \DIFaddbegin \DIFadd{:
}\DIFaddend \begin{equation}
\int_{-\infty}^{\infty}f(x,t)dx = \frac{2ln(e^{\gamma\tau}+1)}{\gamma}
\end{equation}
\DIFdelbegin \DIFdel{which can be solved for $\tau$. For a given fixed total area under the curve, an appropriate value of $\tau$ can be calculated for each value of }\DIFdelend \DIFaddbegin \DIFadd{Thus, ranges defined by different }\DIFaddend $\gamma$ \DIFdelbegin \DIFdel{.
}%DIFDELCMD < 

%DIFDELCMD < %%%
\DIFdel{Thus, }\DIFdelend \DIFaddbegin \DIFadd{values could also result in different range-wide carrying capacities, potentially altering both the ecological (e.g. through stochastic extinction events) and evolutionary (e.g. through the efficiency of selection relative to drift) dynamics of the ranges. Additionally, different combinations of }\DIFaddend $\gamma$ and $\tau$ \DIFdelbegin \DIFdel{are both fixed within a given simulation and $\beta_{t}$ (the location of the center of suitable habitat) is used to simulate climate change. During the periods before and after climate change $\beta_{t}$ is constant , but to simulate climate change it varies with time as follows
}\begin{displaymath}
\DIFdel{\beta_{t}=\nu\eta(t-\hat{t})
}\end{displaymath}
%DIFAUXCMD
\DIFdel{where $\nu$ is the velocity of climate change per generation in terms of discrete patches, $t$ is the current generation, and $\hat{t}$ is the last generation of stable climatic conditions before the onset of climate change. }\DIFdelend \DIFaddbegin \DIFadd{could result in different range widths, which have been shown to impact dispersal evolution within the ranges~\mbox{%DIFAUXCMD
\citep{van1997integrodifference}}%DIFAUXCMD
. To control for these confounding factors, we fixed the range widths for all scenarios and altered $K_{max}$ to maintain a constant range-wide carrying capacity. Specifically, we defined the range width using the $x$ coordinates at which $f(x,t)$ fell below $0.1$ on either side of $\beta_{t}$ and chose $\tau$ and $\gamma$ values for each scenario such that $f(x,t)$ fell below $0.1$ at the same $x$ coordinates (Table A2). We then adjusted $K_{max}$ for each scenario so that the range-wide carrying capacity was constant (Fig. A3).
}\DIFaddend 

\DIFaddbegin [\DIFadd{Figure A3 goes here}]

\DIFaddend \begin{table}
\renewcommand{\arraystretch}{1.5}
  \DIFdelbeginFL %DIFDELCMD < \begin{tabular}{ p{4cm} | p{4cm} | p{1.5cm} | p{1.5cm} | p{1.5cm} }
%DIFDELCMD <     %%%
\DIFdelendFL \DIFaddbeginFL \begin{tabular}{ p{4cm} | p{4cm} | p{1.5cm} | p{1.5cm} | p{1.5cm}  | p{1.5cm} }
    \DIFaddendFL \hline
    \DIFdelbeginFL \DIFdelFL{Habitat gradient at the }\DIFdelendFL \DIFaddbeginFL \DIFaddFL{Starkness of }\DIFaddendFL range edge & \DIFdelbeginFL \DIFdelFL{Potential for local adaptation }\DIFdelendFL \DIFaddbeginFL \DIFaddFL{Slope of niche optimum }\DIFaddendFL & $\gamma$ & $\tau$ & $\lambda$ \DIFaddbeginFL & \DIFaddFL{$K_{max}$ }\DIFaddendFL \\ \hline \hline
     & \DIFdelbeginFL \DIFdelFL{None }\DIFdelendFL \DIFaddbeginFL \DIFaddFL{Flat }\DIFaddendFL & $0.0025$ & \DIFdelbeginFL \DIFdelFL{$250$ }\DIFdelendFL \DIFaddbeginFL \DIFaddFL{$-240$ }\DIFaddendFL & $0$ \DIFaddbeginFL & \DIFaddFL{$240$ }\DIFaddendFL \\
    Shallow & \DIFdelbeginFL \DIFdelFL{Low }\DIFdelendFL \DIFaddbeginFL \DIFaddFL{Shallow }\DIFaddendFL & $0.0025$ & \DIFdelbeginFL \DIFdelFL{$250$ }\DIFdelendFL \DIFaddbeginFL \DIFaddFL{$-240$ }\DIFaddendFL & $0.004$ \DIFaddbeginFL & \DIFaddFL{$240$ }\DIFaddendFL \\
     & \DIFdelbeginFL \DIFdelFL{High }\DIFdelendFL \DIFaddbeginFL \DIFaddFL{Steep }\DIFaddendFL & $0.0025$ & \DIFdelbeginFL \DIFdelFL{$250$ }\DIFdelendFL \DIFaddbeginFL \DIFaddFL{$-240$ }\DIFaddendFL & $0.008$ \DIFaddbeginFL & \DIFaddFL{$240$ }\DIFaddendFL \\ \hline
     & \DIFdelbeginFL \DIFdelFL{None }\DIFdelendFL \DIFaddbeginFL \DIFaddFL{Flat }\DIFaddendFL & \DIFdelbeginFL \DIFdelFL{$0.025$ }\DIFdelendFL \DIFaddbeginFL \DIFaddFL{$0.0075$ }\DIFaddendFL & \DIFdelbeginFL \DIFdelFL{$421.479$ }\DIFdelendFL \DIFaddbeginFL \DIFaddFL{$345.9$ }\DIFaddendFL & $0$ \DIFaddbeginFL & \DIFaddFL{$118.1$ }\DIFaddendFL \\
    Moderate & \DIFdelbeginFL \DIFdelFL{Low }\DIFdelendFL \DIFaddbeginFL \DIFaddFL{Shallow }\DIFaddendFL & \DIFdelbeginFL \DIFdelFL{$0.025$ }\DIFdelendFL \DIFaddbeginFL \DIFaddFL{$0.0075$ }\DIFaddendFL & \DIFdelbeginFL \DIFdelFL{$421.479$ }\DIFdelendFL \DIFaddbeginFL \DIFaddFL{$345.9$ }\DIFaddendFL & $0.004$ \DIFaddbeginFL & \DIFaddFL{$118.1$ }\DIFaddendFL \\
     & \DIFdelbeginFL \DIFdelFL{High }\DIFdelendFL \DIFaddbeginFL \DIFaddFL{Steep }\DIFaddendFL & \DIFdelbeginFL \DIFdelFL{$0.025$ }\DIFdelendFL \DIFaddbeginFL \DIFaddFL{$0.0075$ }\DIFaddendFL & \DIFdelbeginFL \DIFdelFL{$421.479$ }\DIFdelendFL \DIFaddbeginFL \DIFaddFL{$345.9$ }\DIFaddendFL & $0.008$ \DIFaddbeginFL & \DIFaddFL{$118.1$ }\DIFaddendFL \\ \hline
     & \DIFdelbeginFL \DIFdelFL{None }\DIFdelendFL \DIFaddbeginFL \DIFaddFL{Flat }\DIFaddendFL & $0.25$ & \DIFdelbeginFL \DIFdelFL{$421.48$ }\DIFdelendFL \DIFaddbeginFL \DIFaddFL{$630.1$ }\DIFaddendFL & $0$ \DIFaddbeginFL & \DIFaddFL{$66.7$ }\DIFaddendFL \\
    Stark & \DIFdelbeginFL \DIFdelFL{Low }\DIFdelendFL \DIFaddbeginFL \DIFaddFL{Shallow }\DIFaddendFL & $0.25$ & \DIFdelbeginFL \DIFdelFL{$421.48$ }\DIFdelendFL \DIFaddbeginFL \DIFaddFL{$630.1$ }\DIFaddendFL & $0.004$ \DIFaddbeginFL & \DIFaddFL{$66.7$ }\DIFaddendFL \\
     & \DIFdelbeginFL \DIFdelFL{High }\DIFdelendFL \DIFaddbeginFL \DIFaddFL{Steep }\DIFaddendFL & $0.25$ & \DIFdelbeginFL \DIFdelFL{$421.48$ }\DIFdelendFL \DIFaddbeginFL \DIFaddFL{$630.1$ }\DIFaddendFL & $0.008$ \DIFaddbeginFL & \DIFaddFL{$66.7$ }\DIFaddendFL \\ 
    \hline
  \end{tabular}
\caption[LoF entry]{Descriptions and parameter values for the $9$ different experimental scenarios. \DIFaddbeginFL \DIFaddFL{As defined in the text, $\gamma$ affects the starkness of the range boundary, $\tau$ affects the width of the range, $\lambda$ is the slope of the gradient in the niche optimum, and $K_{max}$ is the maximum carrying capacity for patches in the landscape.}\DIFaddendFL }
\label{table:Scenarios}
\end{table}

\DIFdelbegin \paragraph{\DIFdel{Local adaptation}}
%DIFAUXCMD
\addtocounter{paragraph}{-1}%DIFAUXCMD
\DIFdel{To allow an arbitrary degree of local adaptation within the range, the local environmental optima for each patch ($z_{opt,x}$) are set as follows
}\DIFdelend \DIFaddbegin \DIFadd{Thus, $\gamma$ and $\tau$ were both fixed within a given simulation and $\beta_{t}$ (the location of the range center) was used to simulate climate change. During the periods before and after climate change $\beta_{t}$ was constant, but to simulate climate change it varied with time as follows
}\DIFaddend \begin{equation}
\DIFdelbegin \DIFdel{z_{opt,x}=\lambda(x-}\DIFdelend \beta_{t}\DIFaddbegin \DIFadd{=\nu\eta(t-}\hat{t}\DIFaddend )
\end{equation}
where \DIFaddbegin \DIFadd{$\nu$ was the velocity of climate change per generation in terms of discrete patches, $t$ was the current generation, and $\hat{t}$ was the last generation of stable climatic conditions before the onset of climate change.
}

\paragraph{\DIFadd{Environmental niche}}
\DIFadd{The niche optimum ($z_{opt,x}$) varied in space according to
}\begin{equation}
\DIFadd{z_{opt,x}=\lambda(x-\beta_{t})
}\end{equation}
\DIFadd{with }\DIFaddend $\lambda$ \DIFdelbegin \DIFdel{defines the potential for local adaptation with values close to $0$ resulting in little to no change in environmental optima across the range and values of greater magnitude resulting in large differences in environmental optima }\DIFdelend \DIFaddbegin \DIFadd{determining the rate of change in the optimum }\DIFaddend across the range. Individual relative fitness ($w_{i,x}$) values \DIFdelbegin \DIFdel{are }\DIFdelend \DIFaddbegin \DIFadd{were }\DIFaddend then calculated according to the following equation assuming stabilizing selection
\begin{equation}
w_{i,x}=e^{\frac{-(z_{i}-z_{opt,x})^{2}}{2\omega^{2}}}
\end{equation}
where $\omega$ \DIFdelbegin \DIFdel{defines }\DIFdelend \DIFaddbegin \DIFadd{defined }\DIFaddend the strength of stabilizing selection and $z_{i}$ \DIFdelbegin \DIFdel{is }\DIFdelend \DIFaddbegin \DIFadd{was }\DIFaddend an individual's niche phenotype~\citep{lande1976natural}. Thus, an individual's realized fitness \DIFdelbegin \DIFdel{will be }\DIFdelend \DIFaddbegin \DIFadd{was }\DIFaddend higher the closer its niche phenotype ($z_{i}$) \DIFdelbegin \DIFdel{is }\DIFdelend \DIFaddbegin \DIFadd{was }\DIFaddend to the environmental optimum of the patch it \DIFdelbegin \DIFdel{occupies }\DIFdelend \DIFaddbegin \DIFadd{occupied }\DIFaddend ($z_{opt,x}$). All loci \DIFdelbegin \DIFdel{are }\DIFdelend \DIFaddbegin \DIFadd{were }\DIFaddend assumed to contribute additively to an individual's environmental niche value with no dominance or epistasis, meaning an individual's phenotype \DIFdelbegin \DIFdel{is }\DIFdelend \DIFaddbegin \DIFadd{was }\DIFaddend simply the sum of the individual's allele values. \DIFaddbegin \DIFadd{As defined above, $z_{opt,x}$ also shifts with climate change (i.e. with $\beta_{t}$) as would be expected if it corresponded to a phenotypic optimum along a temperature or precipitation gradient within the range~\mbox{%DIFAUXCMD
\citep{davis2001range}}%DIFAUXCMD
. 
}\DIFaddend 

\paragraph{Population dynamics}
Population growth within each patch \DIFdelbegin \DIFdel{is }\DIFdelend \DIFaddbegin \DIFadd{was }\DIFaddend modeled with a stochastic implementation of the classic Ricker model~\citep{ricker1954stock, melbourne2008extinction}. To account for fitness effects on population growth, expected population growth \DIFdelbegin \DIFdel{is }\DIFdelend \DIFaddbegin \DIFadd{was }\DIFaddend scaled by the mean relative fitness of individuals within the patch ($\bar{w_{x}}$) \DIFaddbegin \DIFadd{so that maladaptation resulted in reduced population growth}\DIFaddend . The expected number of new offspring in patch $x$ at time $t+1$ \DIFdelbegin \DIFdel{is then }\DIFdelend \DIFaddbegin \DIFadd{was }\DIFaddend given by
\begin{equation}
\hat{N}_{t+1,x}=\bar{w_{x}}F_{t,x}\frac{R}{\psi}e^{\frac{-RN_{t,x}}{K_{x}}}
\end{equation}
where $F_{t,x}$ \DIFdelbegin \DIFdel{is }\DIFdelend \DIFaddbegin \DIFadd{was }\DIFaddend the number of females in patch $x$ at time $t$, $R$ \DIFdelbegin \DIFdel{is }\DIFdelend \DIFaddbegin \DIFadd{was }\DIFaddend the intrinsic growth rate for the population \DIFaddbegin \DIFadd{and remained constant in both time and space}\DIFaddend , $\psi$ \DIFdelbegin \DIFdel{is }\DIFdelend \DIFaddbegin \DIFadd{was }\DIFaddend the expected sex ratio of the population, $N_{t,x}$ \DIFdelbegin \DIFdel{is }\DIFdelend \DIFaddbegin \DIFadd{was }\DIFaddend the number of individuals (males and females) in patch $x$ at time $t$, and $K_{x}$ \DIFdelbegin \DIFdel{is }\DIFdelend \DIFaddbegin \DIFadd{was }\DIFaddend the local carrying capacity based on the environmental conditions. To incorporate demographic stochasticity, the realized number of offspring for each patch \DIFdelbegin \DIFdel{is }\DIFdelend \DIFaddbegin \DIFadd{was }\DIFaddend then drawn from a Poisson distribution.
\begin{equation}
N_{t+1,x}\sim Poisson(\hat{N}_{t+1,x})
\end{equation}

Parentage of the offspring \DIFdelbegin \DIFdel{is then }\DIFdelend \DIFaddbegin \DIFadd{was }\DIFaddend assigned by random sampling of the local male and female \DIFdelbegin \DIFdel{population }\DIFdelend \DIFaddbegin \DIFadd{populations }\DIFaddend (i.e. polygynandrous mating \DIFaddbegin \DIFadd{assuming a well-mixed population within each patch}\DIFaddend ). The sampling \DIFdelbegin \DIFdel{is }\DIFdelend \DIFaddbegin \DIFadd{was }\DIFaddend weighted by individual fitness and \DIFdelbegin \DIFdel{occurs }\DIFdelend \DIFaddbegin \DIFadd{occurred }\DIFaddend with replacement so highly fit individuals \DIFdelbegin \DIFdel{are }\DIFdelend \DIFaddbegin \DIFadd{were }\DIFaddend likely to have multiple offspring while low fitness individuals \DIFdelbegin \DIFdel{may not have }\DIFdelend \DIFaddbegin \DIFadd{might not have had }\DIFaddend any. Each offspring \DIFdelbegin \DIFdel{inherits }\DIFdelend \DIFaddbegin \DIFadd{inherited }\DIFaddend one allele per locus from each parent, assuming no linkage among loci. After reproduction, all members of the previous generation \DIFdelbegin \DIFdel{die }\DIFdelend \DIFaddbegin \DIFadd{died }\DIFaddend and the offspring \DIFdelbegin \DIFdel{disperse }\DIFdelend \DIFaddbegin \DIFadd{dispersed }\DIFaddend to begin the next generation. \DIFaddbegin \DIFadd{Parameters governing population dynamics (Table A3) were chosen to yield reasonable rates of population growth based on initial exploratory simulations.
}\DIFaddend 

%DIF >  Population dynamics A3
\DIFaddbegin \begin{table}
\renewcommand{\arraystretch}{1.5}
  \begin{tabular}{ p{2cm} | p{8cm} | p{2cm} }
    \hline
    \DIFaddFL{Parameter }& \DIFaddFL{Description }& \DIFaddFL{Value }\\ \hline \hline
    \DIFaddFL{$R$ }& \DIFaddFL{Intrinsic growth rate of the population }& \DIFaddFL{$2$ }\\
    \DIFaddFL{$\psi$ }& \DIFaddFL{Expected sex ratio (females/males) in the population }& \DIFaddFL{$0.5$ }\\
    \DIFaddFL{$\hat{d}$ }& \DIFaddFL{Maximum achievable dispersal phenotype }& \DIFaddFL{$1000$ }\\
    \DIFaddFL{$\rho$ }& \DIFaddFL{Determines the slope of the transition in dispersal phenotypes from $0$ to $D$ }& \DIFaddFL{$0.5$ }\\
    \hline
  \end{tabular}
\caption[LoF entry]{\DIFaddFL{Values and descriptions for parameters related to population growth and dispersal.}}
\label{table:PopPars}
\end{table}

\DIFaddend \paragraph{Mutation}
Inherited alleles \DIFdelbegin \DIFdel{are }\DIFdelend \DIFaddbegin \DIFadd{were }\DIFaddend subject to mutation such that some offspring might not inherit identical copies of certain alleles from their parents. The mutation process \DIFdelbegin \DIFdel{is }\DIFdelend \DIFaddbegin \DIFadd{was }\DIFaddend defined by two parameters for each trait $T$: the diploid mutation rate ($U^{T}$) and the mutational variance ($V_{m}^{T}$). Using these parameters along with the number of loci defining trait $T$ ($L^{T}$), the per locus probability of a mutation \DIFdelbegin \DIFdel{is
}\DIFdelend \DIFaddbegin \DIFadd{was
}\DIFaddend \begin{equation}
\frac{U^{T}}{2L^{T}}
\end{equation}
\DIFdelbegin \DIFdel{Mutational effects are }\DIFdelend \DIFaddbegin \DIFadd{Effect sizes of mutations were }\DIFaddend drawn from a normal distribution with mean $0$ and a standard deviation of
\begin{equation}
\sqrt{V_{m}^{T}U^{T}}
\end{equation}
\DIFdelbegin \DIFdel{By }\DIFdelend \DIFaddbegin \DIFadd{meaning the ratio of small effect versus large effect mutations depended on both $U^{T}$ and $V_{m}^{T}$. We chose parameter values (Table A4) in keeping with previously derived values from the literature~\mbox{%DIFAUXCMD
\citep{gilbert2017local}}%DIFAUXCMD
. For the number of loci used in our simulations, these resulted in mostly mutations of small effect with few large effect mutations. Importantly, by }\DIFaddend defining the mutation process in this \DIFdelbegin \DIFdel{manner rather than setting a }\DIFdelend \DIFaddbegin \DIFadd{way, rather than with a per locus }\DIFaddend probability of mutation and \DIFdelbegin \DIFdel{mutational effect directly, similar mutational dynamics can be imposed }\DIFdelend \DIFaddbegin \DIFadd{a mutation effect size directly, the mutational input per generation was kept constant }\DIFaddend regardless of the number of loci \DIFdelbegin \DIFdel{used in the simulation}\DIFdelend \DIFaddbegin \DIFadd{defining the trait~\mbox{%DIFAUXCMD
\citep{schiffers2014landscape}}%DIFAUXCMD
}\DIFaddend .

%DIF >  Genetics A4
\DIFaddbegin \begin{table}
\renewcommand{\arraystretch}{1.5}
  \begin{tabular}{ p{2cm} | p{8cm} | p{2cm} }
    \hline
    \DIFaddFL{Parameter }& \DIFaddFL{Description }& \DIFaddFL{Value }\\ \hline \hline
    \DIFaddFL{$\omega$ }& \DIFaddFL{Defines the strength of stabilizing selection on fitness traits }& \DIFaddFL{$3$ }\\
    \DIFaddFL{$U^{T}$ }& \DIFaddFL{Diploid mutation rate for each trait }& \DIFaddFL{$0.02$ }\\
    \DIFaddFL{$V_{m}^{T}$ }& \DIFaddFL{Mutational variance for each trait }& \DIFaddFL{$0.0004$ }\\
    \DIFaddFL{$L^{T}$ }& \DIFaddFL{Number of diploid loci defining each trait }& \DIFaddFL{$5$ loci }\\
    \DIFaddFL{$\mu_{1}^{f}$ }& \DIFaddFL{Initial mean allele value for the environmental niche trait }& \DIFaddFL{$0$ }\\
    \DIFaddFL{$\mu_{1}^{d}$ }& \DIFaddFL{Initial mean allele value for the dispersal trait }& \DIFaddFL{$-1$ }\\
    \DIFaddFL{$\sigma_{1}^{f}$ }& \DIFaddFL{Initial standard deviation of allele values for the environmental niche trait }& \DIFaddFL{$0.025$ }\\
    \DIFaddFL{$\sigma_{1}^{d}$ }& \DIFaddFL{Initial standard deviation of allele values for the dispersal trait }& \DIFaddFL{$1$ }\\
    \hline
  \end{tabular}
\caption[LoF entry]{\DIFaddFL{Values and descriptions for parameters defining the genetic components of the model.}}
\label{table:GenPars}
\end{table}

\DIFaddend \paragraph{Dispersal}
Finally, individuals \DIFdelbegin \DIFdel{disperse }\DIFdelend \DIFaddbegin \DIFadd{dispersed }\DIFaddend according to an exponential dispersal kernel defined by each individual's dispersal phenotype. An individual's dispersal phenotype \DIFdelbegin \DIFdel{is }\DIFdelend \DIFaddbegin \DIFadd{was }\DIFaddend the expected dispersal distance and \DIFdelbegin \DIFdel{is }\DIFdelend \DIFaddbegin \DIFadd{was }\DIFaddend given by
\begin{equation}
d_{i} = \frac{\hat{d}\eta e^{\rho\Sigma L^{D}}}{1+e^{\rho\Sigma L^{D}}} 
\end{equation}
where $\hat{d}$ \DIFdelbegin \DIFdel{is }\DIFdelend \DIFaddbegin \DIFadd{was }\DIFaddend the maximum expected dispersal distance in terms of discrete patches, $\rho$ \DIFdelbegin \DIFdel{is }\DIFdelend \DIFaddbegin \DIFadd{was }\DIFaddend a constant determining the slope of the transition between $0$ and $\hat{d}$, and the summation \DIFdelbegin \DIFdel{is }\DIFdelend \DIFaddbegin \DIFadd{was }\DIFaddend taken across all alleles contributing to dispersal. Thus, as with fitness, loci \DIFdelbegin \DIFdel{are }\DIFdelend \DIFaddbegin \DIFadd{were }\DIFaddend assumed to contribute additively with no dominance or epistasis. The expected dispersal distance, $d_{i}$ \DIFdelbegin \DIFdel{is }\DIFdelend \DIFaddbegin \DIFadd{was }\DIFaddend then used to draw a realized distance from an exponential dispersal kernel. \DIFdelbegin \DIFdel{Since the dispersal phenotype is the expected value of the exponential dispersal kernel, it can be used directly to calculate the two dimensional diffusion coefficient of population spread ($D$) . Specifically, since $d_{i}^{2}$ represents the mean squared displacement of an individual with dispersal phenotype $d_{i}$, the two dimensional diffusion coefficient can be calculated as
}\begin{displaymath}
\DIFdel{D = \frac{1}{4}d_{i}^{2}
}\end{displaymath}
%DIFAUXCMD
%DIFDELCMD < 

%DIFDELCMD < %%%
\DIFdel{Once the realized dispersal distance is obtained, the direction of dispersal is }\DIFdelend \DIFaddbegin \DIFadd{The direction of dispersal (in radians) was }\DIFaddend drawn from a uniform distribution bounded by $0$ and $2\pi$. If a dispersal trajectory \DIFdelbegin \DIFdel{takes }\DIFdelend \DIFaddbegin \DIFadd{took }\DIFaddend an individual outside the bounds of the landscape in the $y$ dimension, the individual \DIFdelbegin \DIFdel{reappears }\DIFdelend \DIFaddbegin \DIFadd{reappeared }\DIFaddend at the same $x$ coordinate but the opposite end of the $y$ dimension, thus wrapping the top and bottom edges of the landscape to avoid edge effects. Dispersal \DIFdelbegin \DIFdel{occurs }\DIFdelend \DIFaddbegin \DIFadd{occurred }\DIFaddend from the center of each patch and the individual's new patch \DIFdelbegin \DIFdel{is }\DIFdelend \DIFaddbegin \DIFadd{was }\DIFaddend then determined according to its location in the overlaid grid of $\eta$ x $\eta$ patches (see Figure A1). \DIFaddbegin \DIFadd{Dispersal parameters (Table A3) were chosen to allow a wide range of dispersal phenotypes to evolve in the context of the different experimental scenarios, ranging from highly restrictive to long-distance dispersal.
}\DIFaddend 

\DIFaddbegin \DIFadd{Since the dispersal phenotype was the expected value of the exponential dispersal kernel, it could be used directly to calculate the two dimensional diffusion coefficient of population spread ($D$). Specifically, since $d_{i}^{2}$ represented the mean squared displacement of an individual with dispersal phenotype $d_{i}$, the two dimensional diffusion coefficient could be calculated as
}\begin{equation}
\DIFadd{D = \frac{1}{4}d_{i}^{2}
}\end{equation}
\DIFadd{and subsequently used to calculate the approximate speed of an expansion wave defined by that dispersal phenotype.
}

\DIFaddend \newpage{}

\section*{Appendix B: Supplementary results for varying speeds of climate change}

\renewcommand{\theequation}{B\arabic{equation}}
% redefine the command that creates the equation number.
\renewcommand{\thetable}{B\arabic{table}}
\setcounter{equation}{0}  % reset counter 
\setcounter{figure}{0}
\setcounter{table}{0}

\section*{Extinction probability}
As in the main text, we calculated the cumulative probability of extinction for both slow and fast speeds of climate change here. The figures in this section use the same layout and line types as Figure 2 in the main text to allow for direct comparisons.

[Figures B1\&B2 go here.]

\section*{\DIFdelbegin \DIFdel{Initial }\DIFdelend \DIFaddbegin \DIFadd{Equilibrium }\DIFaddend fitness throughout the landscape}
To assess trends in realized fitness values throughout the landscape, we calculated \DIFdelbegin \DIFdel{the }\DIFdelend patch-level mean individual fitness \DIFdelbegin \DIFdel{value }\DIFdelend for each landscape \DIFdelbegin \DIFdel{just prior to the onset of climate change}\DIFdelend \DIFaddbegin \DIFadd{at equilibrium}\DIFaddend . To simplify the figures, we averaged over the $y$ dimension in which \DIFdelbegin \DIFdel{environmental conditions do not vary. By separating the results for both extinction populationsand those }\DIFdelend \DIFaddbegin \DIFadd{CV in niche genotypes was minimal throughout the range for all scenarios (typically below $0.25$) due to the constant environmental conditions. Populations at the range edge were characterized by reduced fitness compared to core populations, as found in previous models~\mbox{%DIFAUXCMD
\citep{garcia1997genetic}}%DIFAUXCMD
. However, this trend was exacerbated in populations }\DIFaddend that successfully tracked climate change \DIFdelbegin \DIFdel{, we identified a trend in which the edge populations displayed lower fitness in populations that were ultimately successful, particularly in simulations with a gradual environmental gradient at the range edge and }\DIFdelend \DIFaddbegin \DIFadd{compared to those that went extinct, especially in }\DIFaddend the \DIFdelbegin \DIFdel{potential for local adaptation}\DIFdelend \DIFaddbegin \DIFadd{presence of gradual range edges and steep gradients in the niche optimum}\DIFaddend . As realized fitness values do not vary spatially in simulations with no \DIFdelbegin \DIFdel{potential for local adaptation}\DIFdelend \DIFaddbegin \DIFadd{gradient in the niche optimum}\DIFaddend , the following figures only show results for scenarios with a \DIFdelbegin \DIFdel{moderate or high potential for local adaptation}\DIFdelend \DIFaddbegin \DIFadd{shallow or steep gradient}\DIFaddend .

[Figures B3-B5 go here.]

\section*{Dispersal evolution}
Using the same metric of dispersal evolution from the main text (\DIFdelbegin \DIFdel{average change in }\DIFdelend \DIFaddbegin \DIFadd{change in average }\DIFaddend phenotype for each patch), we display here the observed dispersal evolution over the course of climate change for all experimental scenarios. Each histogram in the following figures represents a single experimental scenario as indicated by the figure text. The lower left panel and upper right panel from Figure B7 are the same histograms shown in Figure 3a\&b, but are here placed in the context of all other experimental scenarios.

[Figures B6-B8 go here.]

\section*{\DIFdelbegin \DIFdel{Initial }\DIFdelend \DIFaddbegin \DIFadd{Equilibrium }\DIFaddend dispersal phenotypes}
Here, we present histograms of the \DIFdelbegin \DIFdel{initial }\DIFdelend \DIFaddbegin \DIFadd{equilibrium }\DIFaddend distribution of dispersal phenotypes to demonstrate the importance of those phenotypes in determining population success or extinction during climate change. Dispersal phenotypes are log transformed for easier comparison. As with the dispersal evolution section, each histogram represents a single experimental scenario as indicated by the figure text. Similarly, the lower left panel and upper right panel from Figure B10 are the same histograms shown in Figure 3c\&d, but are here placed in the context of all other experimental scenarios. All histograms additionally have a vertical dashed line indicating the dispersal phenotype necessary to produce an expansion wave traveling at exactly the speed of climate change in each simulation. This value serves as a threshold to distinguish individuals from ultimately successful versus extinct populations.

[Figures B9-B11 go here.]

\newpage{}

%%%%%%%%%%%%%%%%%%%%%
% Bibliography
%%%%%%%%%%%%%%%%%%%%%
% You can either type your references following the examples below, or
% compile your BiBTeX database and paste the contents of your .bbl file
% here. The amnatnat.bst style file should work for this---but please
% let us know if you run into any hitches with it!
% The list below includes sample journal articles, book chapters, and
% Dryad references.

\bibliographystyle{amnat}
\bibliography{main_bib}

\newpage{}

\section*{Figures}

\begin{figure}[h!]
\includegraphics[width=1\textwidth]{"/Users/Topher/Desktop/RangeShifts/ShiftingSlopesOther/SchematicFigures/SimExample"}
\caption{A single example of a simulation with a \DIFdelbeginFL \DIFdelFL{high potential for local adaptation }\DIFdelendFL \DIFaddbeginFL \DIFaddFL{steep gradient in the niche optimum }\DIFaddendFL and a moderate \DIFdelbeginFL \DIFdelFL{habitat gradient defining the }\DIFdelendFL range edge. Information on the (a) abundance, (b) dispersal, and (c) fitness of individuals in each patch is shown for time periods beginning with the last generation of stable climate conditions ($t = 0$) to $40$ generations after the start of climate change. Log transformed mean dispersal phenotypes (b) are shown for each patch. Average patch fitness (c) was calculated based on the mean \DIFdelbeginFL \DIFdelFL{environmental }\DIFdelendFL niche trait of local individuals and the \DIFdelbeginFL \DIFdelFL{environmental }\DIFdelendFL \DIFaddbeginFL \DIFaddFL{niche }\DIFaddendFL optima for each patch.}
\label{fig:SimExample}
\end{figure}

\clearpage

\begin{figure}[h!]
\includegraphics[width=1\textwidth]{"/Users/Topher/Desktop/RangeShifts/ShiftingSlopesOther/ResultFigures/MainExtinction"}
\caption{The cumulative probability of extinction due to \DIFaddbeginFL \DIFaddFL{a moderate speed of }\DIFaddendFL climate change in different experimental scenarios. Graphs show the proportion of simulated populations that went extinct through time for scenarios with \DIFaddbeginFL \DIFaddFL{a }\DIFaddendFL (a) \DIFdelbeginFL \DIFdelFL{no}\DIFdelendFL \DIFaddbeginFL \DIFaddFL{flat}\DIFaddendFL , (b) \DIFdelbeginFL \DIFdelFL{low}\DIFdelendFL \DIFaddbeginFL \DIFaddFL{shallow}\DIFaddendFL , and (c) \DIFdelbeginFL \DIFdelFL{high potential for local adaptation}\DIFdelendFL \DIFaddbeginFL \DIFaddFL{steep gradient in the niche optimum}\DIFaddendFL , and in \DIFdelbeginFL \DIFdelFL{environments }\DIFdelendFL \DIFaddbeginFL \DIFaddFL{ranges }\DIFaddendFL characterized by \DIFdelbeginFL \DIFdelFL{a }\DIFdelendFL shallow (solid line), moderate (dashed line), or stark (dotted line) \DIFdelbeginFL \DIFdelFL{gradient at the range edge}\DIFdelendFL \DIFaddbeginFL \DIFaddFL{edges}\DIFaddendFL .}
\label{fig:ExtProb}
\end{figure}

\clearpage

\begin{figure}[h!]
\includegraphics[width=1\textwidth]{"/Users/Topher/Desktop/RangeShifts/ShiftingSlopesOther/ResultFigures/DispComposite"}
\caption{Patterns in the evolution and the initial distribution of the dispersal trait, highlighting extant simulations \DIFaddbeginFL \DIFaddFL{from a moderate speed of climate change}\DIFaddendFL . Evolution in dispersal (a and b) is shown as the change in the mean dispersal phenotype of each patch from the beginning of the period of climate change to the end. Positive values indicate an increase in average dispersal ability in the patch. \DIFdelbeginFL \DIFdelFL{Initial }\DIFdelendFL \DIFaddbeginFL \DIFaddFL{Equilibrium }\DIFaddendFL distributions of the dispersal trait (c and d) are shown as log transformed dispersal phenotypes of individuals in populations after $2000$ generations of stable climate conditions. In all panels, values associated with extant populations are shown in dark blue. Results are shown for populations with no \DIFdelbeginFL \DIFdelFL{potential for local adaptation }\DIFdelendFL \DIFaddbeginFL \DIFaddFL{gradient in the niche optimum }\DIFaddendFL and a gradual \DIFdelbeginFL \DIFdelFL{environmental gradient at the }\DIFdelendFL range \DIFdelbeginFL \DIFdelFL{boundary }\DIFdelendFL \DIFaddbeginFL \DIFaddFL{edge }\DIFaddendFL (a and c; $n = 155$ extant populations) and for populations with a \DIFdelbeginFL \DIFdelFL{high potential for local adaptation }\DIFdelendFL \DIFaddbeginFL \DIFaddFL{steep gradient in the niche optimum }\DIFaddendFL and a stark \DIFdelbeginFL \DIFdelFL{gradient at the }\DIFdelendFL range \DIFdelbeginFL \DIFdelFL{boundary }\DIFdelendFL \DIFaddbeginFL \DIFaddFL{edge }\DIFaddendFL (b and d; $n = 14$ extant populations). Full results for all parameter combinations are provided in Appendix B.}
\label{fig:Disp}
\end{figure}

\clearpage

\subsection*{Online figures}

\renewcommand{\thefigure}{A\arabic{figure}}
\setcounter{figure}{0}

\begin{figure}[h!]
\includegraphics[width=1\textwidth]{"/Users/Topher/Desktop/RangeShifts/ShiftingSlopesOther/SchematicFigures/f_of_xt"}
\caption{Example \DIFaddbeginFL \DIFaddFL{visualization }\DIFaddendFL of \DIFdelbeginFL \DIFdelFL{the environmentally suitable habitat available to a population, as defined by }\DIFdelendFL $f(x,t)$ in Cartesian space. The parameters of $f(x,t)$ are shown on the figure at significant points along the $x$ axis. \DIFaddbeginFL \DIFaddFL{Specifically, $\beta_{t}$ defined the center of the range, $\gamma$ determined the slope of $f(x,t)$ at the inflection points (i.e. the range edges), and $\tau$ determined the location of the inflection points. }\DIFaddendFL The lattice of discrete \DIFaddbeginFL \DIFaddFL{$\eta$ x $\eta$ }\DIFaddendFL patches in which population dynamics \DIFdelbeginFL \DIFdelFL{occur }\DIFdelendFL \DIFaddbeginFL \DIFaddFL{occurred }\DIFaddendFL is shown beneath. As described in the \textit{Submodels} section of the supplemental materials, $f(x,t)$ \DIFdelbeginFL \DIFdelFL{determines }\DIFdelendFL \DIFaddbeginFL \DIFaddFL{determined }\DIFaddendFL the carrying capacity of the \DIFdelbeginFL \DIFdelFL{discrete $\eta$ x $\eta$ }\DIFdelendFL patches \DIFdelbeginFL \DIFdelFL{. Carrying capacities vary with $f(x,t)$ }\DIFdelendFL along the $x$ dimension of the lattice \DIFdelbeginFL \DIFdelFL{and remain }\DIFdelendFL \DIFaddbeginFL \DIFaddFL{while carrying capacity remained }\DIFaddendFL constant within each column along the $y$ dimension. Landscapes \DIFdelbeginFL \DIFdelFL{are  }\DIFdelendFL \DIFaddbeginFL \DIFaddFL{were }\DIFaddendFL unbounded in the $x$ dimension and implemented with wrapping boundaries in the $y$ dimension.}
\label{Fig:EnvFunction}
\end{figure}

\clearpage

\begin{figure}[h!]
\includegraphics[width=1\textwidth]{"/Users/Topher/Desktop/RangeShifts/ShiftingSlopesOther/SchematicFigures/LifeCycle"}
\caption{The life cycle of simulated populations is shown divided between events contributing to reproduction and dispersal. Each generation \DIFdelbeginFL \DIFdelFL{begins }\DIFdelendFL \DIFaddbeginFL \DIFaddFL{began }\DIFaddendFL with new offspring dispersing according to their phenotype, after which reproduction \DIFdelbeginFL \DIFdelFL{occurs }\DIFdelendFL \DIFaddbeginFL \DIFaddFL{occurred }\DIFaddendFL in local populations defined by the discrete lattice. After reproduction, all parental individuals \DIFdelbeginFL \DIFdelFL{perish}\DIFdelendFL \DIFaddbeginFL \DIFaddFL{perished}\DIFaddendFL , resulting in discrete, non-overlapping generations.}
\label{Fig:LifeCycle}
\end{figure}

\clearpage

\DIFaddbegin \begin{figure}[h!]
\includegraphics[width=1\textwidth]{"/Users/Topher/Desktop/RangeShifts/ShiftingSlopesOther/SchematicFigures/Discretization"}
\caption{\DIFaddFL{The carrying capacity of discrete patches along the $x$ dimension of landscapes. From top to bottom, the plots show the carrying capacities for gradual, moderate, and stark range edges. Points represent the carrying capacity of a discrete $\eta$ x $\eta$ patch in the range. The vertical dashed lines indicate the $x$ coordinates at which $f(x,t)$ declines below $0.1$ and the $\gamma$ value for each plot is listed above.}}
\label{Fig:LifeCycle}
\end{figure}

\clearpage

\DIFaddend \renewcommand{\thefigure}{B\arabic{figure}}
\setcounter{figure}{0}

\begin{figure}[h!]
\includegraphics[width=1\textwidth]{"/Users/Topher/Desktop/RangeShifts/ShiftingSlopesOther/ResultFigures/SlowExtinction"}
\caption{\DIFdelbeginFL \DIFdelFL{Extinction probabilities for }\DIFdelendFL \DIFaddbeginFL \DIFaddFL{The cumulative probability of extinction due to }\DIFaddendFL a slow speed of climate change \DIFaddbeginFL \DIFaddFL{in different experimental scenarios}\DIFaddendFL . Graphs show the proportion of simulated populations that went extinct through time for scenarios with \DIFaddbeginFL \DIFaddFL{a }\DIFaddendFL (a) \DIFdelbeginFL \DIFdelFL{no}\DIFdelendFL \DIFaddbeginFL \DIFaddFL{flat}\DIFaddendFL , (b) \DIFdelbeginFL \DIFdelFL{low}\DIFdelendFL \DIFaddbeginFL \DIFaddFL{shallow}\DIFaddendFL , and (c) \DIFdelbeginFL \DIFdelFL{high potential for local adaptation}\DIFdelendFL \DIFaddbeginFL \DIFaddFL{steep gradient in the niche optimum}\DIFaddendFL , and in \DIFdelbeginFL \DIFdelFL{environments }\DIFdelendFL \DIFaddbeginFL \DIFaddFL{ranges }\DIFaddendFL characterized by a shallow (solid line), moderate (dashed line), or stark (dotted line) \DIFdelbeginFL \DIFdelFL{gradient at the range }\DIFdelendFL edge.}
\label{Fig:ExtProbSlow}
\end{figure}

\clearpage

\begin{figure}[h!]
\includegraphics[width=1\textwidth]{"/Users/Topher/Desktop/RangeShifts/ShiftingSlopesOther/ResultFigures/FastExtinction"}
\caption{\DIFdelbeginFL \DIFdelFL{Extinction probabilities for }\DIFdelendFL \DIFaddbeginFL \DIFaddFL{The cumulative probability of extinction due to }\DIFaddendFL a fast speed of climate change \DIFaddbeginFL \DIFaddFL{in different experimental scenarios}\DIFaddendFL . Graphs show the proportion of simulated populations that went extinct through time for scenarios with \DIFaddbeginFL \DIFaddFL{a }\DIFaddendFL (a) \DIFdelbeginFL \DIFdelFL{no}\DIFdelendFL \DIFaddbeginFL \DIFaddFL{flat}\DIFaddendFL , (b) \DIFdelbeginFL \DIFdelFL{low}\DIFdelendFL \DIFaddbeginFL \DIFaddFL{shallow}\DIFaddendFL , and (c) \DIFdelbeginFL \DIFdelFL{high potential for local adaptation}\DIFdelendFL \DIFaddbeginFL \DIFaddFL{steep gradient in the niche optimum}\DIFaddendFL , and in \DIFdelbeginFL \DIFdelFL{environments }\DIFdelendFL \DIFaddbeginFL \DIFaddFL{ranges }\DIFaddendFL characterized by a shallow (solid line), moderate (dashed line), or stark (dotted line) \DIFdelbeginFL \DIFdelFL{gradient at the range }\DIFdelendFL edge.}
\label{Fig:ExtProbFast}
\end{figure}

\clearpage

\begin{figure}[h!]
\includegraphics[width=1\textwidth]{"/Users/Topher/Desktop/RangeShifts/ShiftingSlopesOther/ResultFigures/SlowInitFitSpace"}
\caption{Individual fitness along the $x$ \DIFdelbeginFL \DIFdelFL{dimensions }\DIFdelendFL \DIFaddbeginFL \DIFaddFL{dimension }\DIFaddendFL of the landscape \DIFdelbeginFL \DIFdelFL{prior to the onset of climate change}\DIFdelendFL \DIFaddbeginFL \DIFaddFL{at equilibrium}\DIFaddendFL . Points represent the mean across simulations and error bars are interquartile ranges. Population status (extinct or successful) was determined for a slow speed of climate change.}
\label{Fig:InitFitSlow}
\end{figure}

\clearpage

\begin{figure}[h!]
\includegraphics[width=1\textwidth]{"/Users/Topher/Desktop/RangeShifts/ShiftingSlopesOther/ResultFigures/MainInitFitSpace"}
\caption{Individual fitness along the $x$ \DIFdelbeginFL \DIFdelFL{dimensions }\DIFdelendFL \DIFaddbeginFL \DIFaddFL{dimension }\DIFaddendFL of the landscape \DIFdelbeginFL \DIFdelFL{prior to the onset of climate change}\DIFdelendFL \DIFaddbeginFL \DIFaddFL{at equilibrium}\DIFaddendFL . Points represent the mean across simulations and error bars are interquartile ranges. Population status (extinct or successful) was determined for a moderate speed of climate change.}
\label{Fig:InitFit}
\end{figure}

\clearpage

\begin{figure}[h!]
\includegraphics[width=1\textwidth]{"/Users/Topher/Desktop/RangeShifts/ShiftingSlopesOther/ResultFigures/FastInitFitSpace"}
\caption{Individual fitness along the $x$ \DIFdelbeginFL \DIFdelFL{dimensions }\DIFdelendFL \DIFaddbeginFL \DIFaddFL{dimension }\DIFaddendFL of the landscape \DIFdelbeginFL \DIFdelFL{prior to the onset of climate change}\DIFdelendFL \DIFaddbeginFL \DIFaddFL{at equilibrium}\DIFaddendFL . Points represent the mean across simulations and error bars are interquartile ranges. Population status (extinct or successful) was determined for a fast speed of climate change.}
\label{Fig:InitFitFast}
\end{figure}

\clearpage

\begin{figure}[h!]
\includegraphics[width=1\textwidth]{"/Users/Topher/Desktop/RangeShifts/ShiftingSlopesOther/ResultFigures/SlowDispEvol"}
\caption{Observed dispersal evolution in populations responding to a slow speed of climate change. Positive values indicate an increase in average dispersal ability \DIFdelbeginFL \DIFdelFL{over the course of }\DIFdelendFL \DIFaddbeginFL \DIFaddFL{during }\DIFaddendFL climate change. The values \DIFdelbeginFL \DIFdelFL{associate }\DIFdelendFL \DIFaddbeginFL \DIFaddFL{associated }\DIFaddendFL with populations successfully tracking climate change are shown in dark blue and the total number of surviving populations is indicated in the top left corner. The experimental scenario corresponding to each histogram is indicated on the figure.}
\label{Fig:DispEvolSlow}
\end{figure}

\clearpage

\begin{figure}[h!]
\includegraphics[width=1\textwidth]{"/Users/Topher/Desktop/RangeShifts/ShiftingSlopesOther/ResultFigures/MainDispEvol"}
\caption{Observed dispersal evolution in populations responding to a moderate speed of climate change. Positive values indicate an increase in average dispersal ability \DIFdelbeginFL \DIFdelFL{over the course of }\DIFdelendFL \DIFaddbeginFL \DIFaddFL{during }\DIFaddendFL climate change. The values \DIFdelbeginFL \DIFdelFL{associate }\DIFdelendFL \DIFaddbeginFL \DIFaddFL{associated }\DIFaddendFL with populations successfully tracking climate change are shown in dark blue and the total number of surviving populations is indicated in the top left corner. The experimental scenario corresponding to each histogram is indicated on the figure.}
\label{Fig:DispEvolMain}
\end{figure}

\clearpage

\begin{figure}[h!]
\includegraphics[width=1\textwidth]{"/Users/Topher/Desktop/RangeShifts/ShiftingSlopesOther/ResultFigures/FastDispEvol"}
\caption{Observed dispersal evolution in populations responding to a fast speed of climate change. Positive values indicate an increase in average dispersal ability \DIFdelbeginFL \DIFdelFL{over the course of }\DIFdelendFL \DIFaddbeginFL \DIFaddFL{during }\DIFaddendFL climate change. The values \DIFdelbeginFL \DIFdelFL{associate }\DIFdelendFL \DIFaddbeginFL \DIFaddFL{associated }\DIFaddendFL with populations successfully tracking climate change are shown in dark blue and the total number of surviving populations is indicated in the top left corner. The experimental scenario corresponding to each histogram is indicated on the figure.}
\label{Fig:DispEvolFast}
\end{figure}

\clearpage

\begin{figure}[h!]
\includegraphics[width=1\textwidth]{"/Users/Topher/Desktop/RangeShifts/ShiftingSlopesOther/ResultFigures/SlowInitDispVals"}
\caption{Distributions of the dispersal phenotypes observed in \DIFaddbeginFL \DIFaddFL{equilibrium }\DIFaddendFL populations\DIFdelbeginFL \DIFdelFL{just prior to the onset of climate change}\DIFdelendFL . Phenotypes associated with populations that ultimately survived climate change are shown in dark blue and the total number of surviving populations is indicated in the top left corner. Vertical dashed lines indicate the dispersal phenotype necessary to produce an expansion wave exactly matching a slow speed of climate change.}
\label{Fig:InitDispSlow}
\end{figure}

\clearpage

\begin{figure}[h!]
\includegraphics[width=1\textwidth]{"/Users/Topher/Desktop/RangeShifts/ShiftingSlopesOther/ResultFigures/MainInitDispVals"}
\caption{Distributions of the dispersal phenotypes observed in \DIFaddbeginFL \DIFaddFL{equilibrium }\DIFaddendFL populations\DIFdelbeginFL \DIFdelFL{just prior to the onset of climate change}\DIFdelendFL . Phenotypes associated with populations that ultimately survived climate change are shown in dark blue and the total number of surviving populations is indicated in the top left corner. Vertical dashed lines indicate the dispersal phenotype necessary to produce an expansion wave exactly matching a moderate speed of climate change.}
\label{Fig:InitDispMain}
\end{figure}

\clearpage

\begin{figure}[h!]
\includegraphics[width=1\textwidth]{"/Users/Topher/Desktop/RangeShifts/ShiftingSlopesOther/ResultFigures/FastInitDispVals"}
\caption{Distributions of the dispersal phenotypes observed in \DIFaddbeginFL \DIFaddFL{equilibrium }\DIFaddendFL populations\DIFdelbeginFL \DIFdelFL{just prior to the onset of climate change}\DIFdelendFL . Phenotypes associated with populations that ultimately survived climate change are shown in dark blue and the total number of surviving populations is indicated in the top left corner. Vertical dashed lines indicate the dispersal phenotype necessary to produce an expansion wave exactly matching a fast speed of climate change.}
\label{Fig:InitDispFast}
\end{figure}

\end{document}
